% Options for packages loaded elsewhere
\PassOptionsToPackage{unicode,pdfpagelabels}{hyperref}
\PassOptionsToPackage{hyphens}{url}
\PassOptionsToPackage{dvipsnames,svgnames,x11names}{xcolor}
%
\documentclass[
  12pt,
  a4paper,
]{report}

\usepackage{amsmath,amssymb}
\usepackage{iftex}
\ifPDFTeX
  \usepackage[T1]{fontenc}
  \usepackage[utf8]{inputenc}
  \usepackage{textcomp} % provide euro and other symbols
\else % if luatex or xetex
  \usepackage{unicode-math}
  \defaultfontfeatures{Scale=MatchLowercase}
  \defaultfontfeatures[\rmfamily]{Ligatures=TeX,Scale=1}
\fi
\usepackage{lmodern}
\ifPDFTeX\else  
    % xetex/luatex font selection
\fi
% Use upquote if available, for straight quotes in verbatim environments
\IfFileExists{upquote.sty}{\usepackage{upquote}}{}
\IfFileExists{microtype.sty}{% use microtype if available
  \usepackage[]{microtype}
  \UseMicrotypeSet[protrusion]{basicmath} % disable protrusion for tt fonts
}{}
\makeatletter
\@ifundefined{KOMAClassName}{% if non-KOMA class
  \IfFileExists{parskip.sty}{%
    \usepackage{parskip}
  }{% else
    \setlength{\parindent}{0pt}
    \setlength{\parskip}{6pt plus 2pt minus 1pt}}
}{% if KOMA class
  \KOMAoptions{parskip=half}}
\makeatother
\usepackage{xcolor}
\usepackage[top=2cm,bottom=2cm,left=1.5cm,right=1.5cm,marginparsep=2cm]{geometry}
\setlength{\emergencystretch}{3em} % prevent overfull lines
\setcounter{secnumdepth}{5}
% Make \paragraph and \subparagraph free-standing
\makeatletter
\ifx\paragraph\undefined\else
  \let\oldparagraph\paragraph
  \renewcommand{\paragraph}{
    \@ifstar
      \xxxParagraphStar
      \xxxParagraphNoStar
  }
  \newcommand{\xxxParagraphStar}[1]{\oldparagraph*{#1}\mbox{}}
  \newcommand{\xxxParagraphNoStar}[1]{\oldparagraph{#1}\mbox{}}
\fi
\ifx\subparagraph\undefined\else
  \let\oldsubparagraph\subparagraph
  \renewcommand{\subparagraph}{
    \@ifstar
      \xxxSubParagraphStar
      \xxxSubParagraphNoStar
  }
  \newcommand{\xxxSubParagraphStar}[1]{\oldsubparagraph*{#1}\mbox{}}
  \newcommand{\xxxSubParagraphNoStar}[1]{\oldsubparagraph{#1}\mbox{}}
\fi
\makeatother


\providecommand{\tightlist}{%
  \setlength{\itemsep}{0pt}\setlength{\parskip}{0pt}}\usepackage{longtable,booktabs,array}
\usepackage{calc} % for calculating minipage widths
% Correct order of tables after \paragraph or \subparagraph
\usepackage{etoolbox}
\makeatletter
\patchcmd\longtable{\par}{\if@noskipsec\mbox{}\fi\par}{}{}
\makeatother
% Allow footnotes in longtable head/foot
\IfFileExists{footnotehyper.sty}{\usepackage{footnotehyper}}{\usepackage{footnote}}
\makesavenoteenv{longtable}
\usepackage{graphicx}
\makeatletter
\def\maxwidth{\ifdim\Gin@nat@width>\linewidth\linewidth\else\Gin@nat@width\fi}
\def\maxheight{\ifdim\Gin@nat@height>\textheight\textheight\else\Gin@nat@height\fi}
\makeatother
% Scale images if necessary, so that they will not overflow the page
% margins by default, and it is still possible to overwrite the defaults
% using explicit options in \includegraphics[width, height, ...]{}
\setkeys{Gin}{width=\maxwidth,height=\maxheight,keepaspectratio}
% Set default figure placement to htbp
\makeatletter
\def\fps@figure{htbp}
\makeatother

\makeatletter
\@ifpackageloaded{bookmark}{}{\usepackage{bookmark}}
\makeatother
\makeatletter
\@ifpackageloaded{caption}{}{\usepackage{caption}}
\AtBeginDocument{%
\ifdefined\contentsname
  \renewcommand*\contentsname{Table of contents}
\else
  \newcommand\contentsname{Table of contents}
\fi
\ifdefined\listfigurename
  \renewcommand*\listfigurename{List of Figures}
\else
  \newcommand\listfigurename{List of Figures}
\fi
\ifdefined\listtablename
  \renewcommand*\listtablename{List of Tables}
\else
  \newcommand\listtablename{List of Tables}
\fi
\ifdefined\figurename
  \renewcommand*\figurename{Figure}
\else
  \newcommand\figurename{Figure}
\fi
\ifdefined\tablename
  \renewcommand*\tablename{Table}
\else
  \newcommand\tablename{Table}
\fi
}
\@ifpackageloaded{float}{}{\usepackage{float}}
\floatstyle{ruled}
\@ifundefined{c@chapter}{\newfloat{codelisting}{h}{lop}}{\newfloat{codelisting}{h}{lop}[chapter]}
\floatname{codelisting}{Listing}
\newcommand*\listoflistings{\listof{codelisting}{List of Listings}}
\makeatother
\makeatletter
\makeatother
\makeatletter
\@ifpackageloaded{caption}{}{\usepackage{caption}}
\@ifpackageloaded{subcaption}{}{\usepackage{subcaption}}
\makeatother

\ifLuaTeX
  \usepackage{selnolig}  % disable illegal ligatures
\fi
\usepackage{bookmark}

\IfFileExists{xurl.sty}{\usepackage{xurl}}{} % add URL line breaks if available
\urlstyle{same} % disable monospaced font for URLs
\hypersetup{
  pdftitle={COMPSYS 304: Computer Architecture},
  pdfauthor={Nicholas Russell},
  colorlinks=true,
  linkcolor={blue},
  filecolor={Maroon},
  citecolor={Blue},
  urlcolor={Blue},
  pdfcreator={LaTeX via pandoc}}


\title{COMPSYS 304: Computer Architecture}
\usepackage{etoolbox}
\makeatletter
\providecommand{\subtitle}[1]{% add subtitle to \maketitle
  \apptocmd{\@title}{\par {\large #1 \par}}{}{}
}
\makeatother
\subtitle{Lecture Notes}
\author{Nicholas Russell}
\date{2024-03-09}

\begin{document}
\maketitle

\renewcommand*\contentsname{Table of contents}
{
\hypersetup{linkcolor=}
\setcounter{tocdepth}{2}
\tableofcontents
}

\bookmarksetup{startatroot}

\chapter{Introduction}\label{introduction}

These notes are compiled from the lectures of COMPSYS 304. They are
intended as a personal reference to help with assignments, exam
preparation, and understanding key concepts in Computer Architecture.

The notes are organized by lecture and cover a wide range of topics from
basic computer architecture principles to more advanced subjects like
MIPS implementation and performance analysis. Feel free to add personal
insights, additional readings, or questions as you review this material.

\section{Organization of the Notes}\label{organization-of-the-notes}

The notes are divided by lecture content, with each section
corresponding to a specific set of topics. Here's how they're organized:

\begin{itemize}
\item
  \textbf{Lecture 1-3: Basics of Computer Architecture}\\
  Key topics include Instruction Set Architecture (ISA), memory
  hierarchy, and basic CPU organization.
\item
  \textbf{Lecture 4-6: MIPS Architecture}\\
  Covers the MIPS instruction set, control flow, and subroutine
  handling.
\item
  \textbf{Lecture 7-9: CPU Implementation}\\
  Focuses on different methods for implementing CPUs and the trade-offs
  involved.
\item
  \textbf{Lecture 10-12: Digital Circuits and Datapath Design}\\
  Reviews fundamental digital circuit concepts and discusses the design
  of a MIPS datapath.
\item
  \textbf{Lecture 13-15: Performance Analysis}\\
  Provides an in-depth look at how different design choices impact CPU
  performance.
\end{itemize}

\section{How I Use These Notes}\label{how-i-use-these-notes}

These notes are a living document, and I intend to update them as I gain
a deeper understanding of the material. Here's how I use them:

\begin{itemize}
\tightlist
\item
  \textbf{Quick Reference}: For quick lookups, the Table of Contents
  will help me jump directly to the relevant section.
\item
  \textbf{In-Depth Study}: For exam preparation, I'll revisit each
  section, ensuring I understand each concept before moving on.
\item
  \textbf{Personal Insights}: I'll be adding my own thoughts, additional
  notes from readings, and potential questions for further study.
\end{itemize}

I might also include exercises or practical examples that help solidify
my understanding of the more complex topics.

\part{Lectures 1 - 3}

\chapter{Introduction \& Course
Overview}\label{introduction-course-overview}

This chapter provides an introduction to the course, some background,
and an overview of the course itself.

\section{Background}\label{background}

In the last six decades, computer technology has made incredible
progress due to innovations in both semiconductor technology and
computer architecture.

What do we mean from ``Performance''?

The relative performance can be measured by standard benchmarks, which
depend on the specific target applications.

Performance depends on clock frequency and other factors, as discussed
more in later chapters.

Intel X86 Processor from 1978 to 2018: - Intel 8086 (1978): 1 core, 1 W,
5 - 10 MHz - Core i7-8086K (2018): 6 cores, 95 W, 4 GHz

\section{Course Details}\label{course-details}

\subsection{Learning outcomes}\label{learning-outcomes}

The main learning outcomes of this course are: - To understand the
basics of modern computer architectures and quantative principles of
computer design in order to develop a conceptual understanding of issues
involved in designing a high performance computer system. - To use and
apply this knowledge to design computer systems or select computers for
specific tasks. This course will give you an understanding of the
effects of design decisions on system performance and makes you a
well-informed consumer in addition to a processor designer.

Recommended Textbook: - David A. Patterson and John L. Hennessey,
Computer Organization and Design: The Hardware/Software Interface, Fifth
edition, 2013 by Elsevier/Morgan Kaufmann Publishers (or 3rd or 4th
editions). - Lecture notes provided on canvas (these will be summarised
in these notes).

\subsection{Course Overview (learning
outcomes)}\label{course-overview-learning-outcomes}

Part 1: in this part you will learn to: - Design and evaluate the
instruction set architectures (both RISC and CISC) and how it can be
related to the hardware/software interface in a computer system with a
quick review of assembly programming. - Understand different processor
implementation methods including the basic single-cycle implementation
and how it can be extended to a multi-cycle, pipelined and superscalar
implementations. - Understand performance evaluation techniques and
their relation to the target applications and the processor workload.

Part 2: in this part you will learn to: - Understand the memory
hierarchy in a modern computer system and its impact on the performance
of the system. This includes physical and virtual memory systems and
basics of cache memories. - Understand some basic principles of parallel
computing using special topics in this course (more advanced materials
covered in some elective courses).

\subsection{Assessment}\label{assessment}

Three assignments and one test in addition to the final exam. The first
assignment is mainly on the instruction set architecture design,
hardware/software interfacing and review of assembly programming. The
second assignment is related to processor implementation and performance
issues. The third assignment is related to memory hierarchy system and
multiprocessing.

The test only covers the first part of the course. The final exam covers
the whole course. - Assignment 1: 8\% due Fri. 9 August. - Assignment 2:
7\% due Fri. 23 August. - Assignment 3: 15\% due Fri. 4 October. - Test:
20\% (in week 7, Wednesday 11 September) - Final exam: 50\%

\chapter{}\label{section}




\end{document}
