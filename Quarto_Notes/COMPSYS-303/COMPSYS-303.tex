% Options for packages loaded elsewhere
\PassOptionsToPackage{unicode,pdfpagelabels}{hyperref}
\PassOptionsToPackage{hyphens}{url}
\PassOptionsToPackage{dvipsnames,svgnames,x11names}{xcolor}
%
\documentclass[
  14pt,
  a4paper,
  numbers=noendperiod,
  headinclude=true,
  footinclude=true,
  DIV=calc]{scrreprt}

\usepackage{amsmath,amssymb}
\usepackage{iftex}
\ifPDFTeX
  \usepackage[T1]{fontenc}
  \usepackage[utf8]{inputenc}
  \usepackage{textcomp} % provide euro and other symbols
\else % if luatex or xetex
  \usepackage{unicode-math}
  \defaultfontfeatures{Scale=MatchLowercase}
  \defaultfontfeatures[\rmfamily]{Ligatures=TeX,Scale=1}
\fi
\usepackage[]{lmodern}
\ifPDFTeX\else  
    % xetex/luatex font selection
\fi
% Use upquote if available, for straight quotes in verbatim environments
\IfFileExists{upquote.sty}{\usepackage{upquote}}{}
\IfFileExists{microtype.sty}{% use microtype if available
  \usepackage[]{microtype}
  \UseMicrotypeSet[protrusion]{basicmath} % disable protrusion for tt fonts
}{}
\makeatletter
\@ifundefined{KOMAClassName}{% if non-KOMA class
  \IfFileExists{parskip.sty}{%
    \usepackage{parskip}
  }{% else
    \setlength{\parindent}{0pt}
    \setlength{\parskip}{6pt plus 2pt minus 1pt}}
}{% if KOMA class
  \KOMAoptions{parskip=half}}
\makeatother
\usepackage{xcolor}
\usepackage[top=3cm,bottom=3cm,left=1.5cm,right=1.5cm]{geometry}
\setlength{\emergencystretch}{3em} % prevent overfull lines
\setcounter{secnumdepth}{5}
% Make \paragraph and \subparagraph free-standing
\makeatletter
\ifx\paragraph\undefined\else
  \let\oldparagraph\paragraph
  \renewcommand{\paragraph}{
    \@ifstar
      \xxxParagraphStar
      \xxxParagraphNoStar
  }
  \newcommand{\xxxParagraphStar}[1]{\oldparagraph*{#1}\mbox{}}
  \newcommand{\xxxParagraphNoStar}[1]{\oldparagraph{#1}\mbox{}}
\fi
\ifx\subparagraph\undefined\else
  \let\oldsubparagraph\subparagraph
  \renewcommand{\subparagraph}{
    \@ifstar
      \xxxSubParagraphStar
      \xxxSubParagraphNoStar
  }
  \newcommand{\xxxSubParagraphStar}[1]{\oldsubparagraph*{#1}\mbox{}}
  \newcommand{\xxxSubParagraphNoStar}[1]{\oldsubparagraph{#1}\mbox{}}
\fi
\makeatother
\pagestyle{plain}


\providecommand{\tightlist}{%
  \setlength{\itemsep}{0pt}\setlength{\parskip}{0pt}}\usepackage{longtable,booktabs,array}
\usepackage{calc} % for calculating minipage widths
% Correct order of tables after \paragraph or \subparagraph
\usepackage{etoolbox}
\makeatletter
\patchcmd\longtable{\par}{\if@noskipsec\mbox{}\fi\par}{}{}
\makeatother
% Allow footnotes in longtable head/foot
\IfFileExists{footnotehyper.sty}{\usepackage{footnotehyper}}{\usepackage{footnote}}
\makesavenoteenv{longtable}
\usepackage{graphicx}
\makeatletter
\def\maxwidth{\ifdim\Gin@nat@width>\linewidth\linewidth\else\Gin@nat@width\fi}
\def\maxheight{\ifdim\Gin@nat@height>\textheight\textheight\else\Gin@nat@height\fi}
\makeatother
% Scale images if necessary, so that they will not overflow the page
% margins by default, and it is still possible to overwrite the defaults
% using explicit options in \includegraphics[width, height, ...]{}
\setkeys{Gin}{width=\maxwidth,height=\maxheight,keepaspectratio}
% Set default figure placement to htbp
\makeatletter
\def\fps@figure{htbp}
\makeatother

\usepackage{graphicx}
\usepackage{siunitx}
\usepackage{booktabs}
\usepackage{xcolor}
\usepackage{float}
\usepackage{fancyhdr}
\usepackage[showdow,en-GB]{datetime2}
\DTMlangsetup[en-GB]{ord=raise,monthyearsep={,\space}}

% \usepackage{fontspec}
\sisetup{per-mode = symbol} % Set SI unit fractions to use a solidus

\renewcommand*{\chapterformat}{\textbf{Lecture \thechapter: \;}}

\setkomafont{disposition}{\normalcolor\bfseries}
\setkomafont{section}{\normalfont\Large\bfseries}
\setkomafont{chapter}{\normalfont\LARGE\bfseries}

\RedeclareSectionCommand[
  beforeskip=10pt, % Space before the chapter
  afterskip=20pt   % Space after the chapter
]{chapter}

\fancypagestyle{plain}{%
  \fancyhf{}%
  \fancyhead[L]{\textbf{Nicholas Russell}}%
  \fancyhead[C]{COMPSYS 303 Notes}%
  \fancyhead[R]{\DTMDate{2024-10-05}}%
  \fancyfoot[R]{\thepage}%
  \renewcommand{\headrulewidth}{0.4pt}%
  \renewcommand{\footrulewidth}{0pt}%
}
\pagestyle{plain}
\setlength{\headheight}{16.5pt}
\KOMAoption{captions}{tableheading}
\makeatletter
\@ifpackageloaded{tcolorbox}{}{\usepackage[skins,breakable]{tcolorbox}}
\@ifpackageloaded{fontawesome5}{}{\usepackage{fontawesome5}}
\definecolor{quarto-callout-color}{HTML}{909090}
\definecolor{quarto-callout-note-color}{HTML}{0758E5}
\definecolor{quarto-callout-important-color}{HTML}{CC1914}
\definecolor{quarto-callout-warning-color}{HTML}{EB9113}
\definecolor{quarto-callout-tip-color}{HTML}{00A047}
\definecolor{quarto-callout-caution-color}{HTML}{FC5300}
\definecolor{quarto-callout-color-frame}{HTML}{acacac}
\definecolor{quarto-callout-note-color-frame}{HTML}{4582ec}
\definecolor{quarto-callout-important-color-frame}{HTML}{d9534f}
\definecolor{quarto-callout-warning-color-frame}{HTML}{f0ad4e}
\definecolor{quarto-callout-tip-color-frame}{HTML}{02b875}
\definecolor{quarto-callout-caution-color-frame}{HTML}{fd7e14}
\makeatother
\makeatletter
\@ifpackageloaded{bookmark}{}{\usepackage{bookmark}}
\makeatother
\makeatletter
\@ifpackageloaded{caption}{}{\usepackage{caption}}
\AtBeginDocument{%
\ifdefined\contentsname
  \renewcommand*\contentsname{Table of contents}
\else
  \newcommand\contentsname{Table of contents}
\fi
\ifdefined\listfigurename
  \renewcommand*\listfigurename{List of Figures}
\else
  \newcommand\listfigurename{List of Figures}
\fi
\ifdefined\listtablename
  \renewcommand*\listtablename{List of Tables}
\else
  \newcommand\listtablename{List of Tables}
\fi
\ifdefined\figurename
  \renewcommand*\figurename{Figure}
\else
  \newcommand\figurename{Figure}
\fi
\ifdefined\tablename
  \renewcommand*\tablename{Table}
\else
  \newcommand\tablename{Table}
\fi
}
\@ifpackageloaded{float}{}{\usepackage{float}}
\floatstyle{ruled}
\@ifundefined{c@chapter}{\newfloat{codelisting}{h}{lop}}{\newfloat{codelisting}{h}{lop}[chapter]}
\floatname{codelisting}{Listing}
\newcommand*\listoflistings{\listof{codelisting}{List of Listings}}
\makeatother
\makeatletter
\makeatother
\makeatletter
\@ifpackageloaded{caption}{}{\usepackage{caption}}
\@ifpackageloaded{subcaption}{}{\usepackage{subcaption}}
\makeatother

\ifLuaTeX
  \usepackage{selnolig}  % disable illegal ligatures
\fi
\usepackage{bookmark}

\IfFileExists{xurl.sty}{\usepackage{xurl}}{} % add URL line breaks if available
\urlstyle{same} % disable monospaced font for URLs
\hypersetup{
  pdftitle={COMPSYS 303: Microcomputers and Embedded Systems},
  pdfauthor={Nicholas Russell},
  colorlinks=true,
  linkcolor={blue},
  filecolor={Maroon},
  citecolor={Blue},
  urlcolor={Blue},
  pdfcreator={LaTeX via pandoc}}


\title{COMPSYS 303: Microcomputers and Embedded Systems}
\usepackage{etoolbox}
\makeatletter
\providecommand{\subtitle}[1]{% add subtitle to \maketitle
  \apptocmd{\@title}{\par {\large #1 \par}}{}{}
}
\makeatother
\subtitle{Lecture Notes}
\author{Nicholas Russell}
\date{Saturday, October 5, 2024}

\begin{document}
\maketitle

\renewcommand*\contentsname{Table of contents}
{
\hypersetup{linkcolor=}
\setcounter{tocdepth}{2}
\tableofcontents
}

\bookmarksetup{startatroot}

\chapter*{Introduction}\label{introduction}

\markboth{Introduction}{Introduction}

These notes provide a detailed breakdown of the lectures for COMPSYS
303. Each lecture is represented in its own document, allowing for
modular learning and easy reference.

\section*{Lectures Included}\label{lectures-included}

\markright{Lectures Included}

\textbf{Lecture 1: Concurrency} Overview of concurrency in embedded
systems, challenges like race conditions and priority inversion, and
real-world examples.

\textbf{Lecture 2: SCCharts and Synchronous Concurrency} Detailed
discussion on SCCharts, the synchronous approach, and cyber-physical
systems.

\textbf{Lecture 3: Peripherals and Interfacing} Types of I/O, data
transfer mechanisms, and interrupt handling in embedded systems.

\textbf{Lecture 4: Embedded Processors} Overview of processor design in
embedded systems, FSMD, types of processors, and synthesis from C code.

\textbf{Lecture 5: SoC and SOPC Buses} Discussion of bus design issues,
synchronous vs.~asynchronous buses, and bus arbitration.

\textbf{Lecture 6: Embedded Control Systems 1} Real-time systems,
sampling and quantization, aliasing, and worst-case execution time.

\textbf{Lecture 7: Embedded Control Systems 2} Interaction with
real-world peripherals, including PWM, stepper motors, and ADC.

\textbf{Lecture 8: Embedded Control Systems 3} Control of physical
processes, key metrics in control systems, and examples like cruise
control.

\textbf{Lecture 9: Industrial Automation 1} Introduction to industrial
automation, PLCs, HMIs, and generations of automation.

\textbf{Lecture 10: Industrial Automation 2} IEC 61131 standard, ladder
logic features, and an example of a pneumatic cylinder control sequence.

\part{Partha's Lectures}

\chapter{Concurrency}\label{concurrency}

\section{Introduction to Embedded Systems and
Concurrency}\label{introduction-to-embedded-systems-and-concurrency}

Embedded systems are a specialized class of computers designed to
continuously interact with their environment in a real-time, often
reactive manner. Unlike general-purpose computers, embedded systems are
purpose-built to handle specific tasks under stringent conditions,
including timing constraints. These systems are embedded as integral
parts of larger devices and must operate dependably, often without human
intervention. Examples include automotive control systems, medical
devices like pacemakers, and consumer electronics such as washing
machines.

When deadlines must be met without exception, the system is termed a
\textbf{hard real-time system}; otherwise, it is a \textbf{soft
real-time system}. In hard real-time systems, any missed deadline could
result in catastrophic failure, such as in medical devices or automotive
safety features. In contrast, soft real-time systems can tolerate
occasional missed deadlines, although they may affect performance.

The concept of \textbf{concurrency} is crucial in embedded systems
because these systems often involve multiple processes interacting
simultaneously, such as a pacemaker continuously monitoring the heart
and delivering pulses when needed. Concurrency enables these systems to
effectively manage different tasks that need to be performed either
synchronously or asynchronously. For example, while monitoring the
heart, a pacemaker must also manage pulse delivery in real-time,
demonstrating the need for concurrent tasks.

\section{Key Concepts in Concurrency}\label{key-concepts-in-concurrency}

\textbf{Concurrency} refers to multiple components or processes
operating at the same time within a system. In embedded systems like
pacemakers, various components (controllers, sensors, and timers) work
concurrently to ensure precise timing between events. Concurrency
becomes especially challenging when different tasks access shared
resources simultaneously, potentially leading to issues such as
\textbf{race conditions}, \textbf{deadlocks}, and \textbf{livelocks}.

\begin{itemize}
\item
  \textbf{Race Conditions:} These occur when two or more processes
  access shared data concurrently, and the final result depends on the
  timing of their access. A typical example is the producer-consumer
  problem, where one process generates data while another consumes it.
  If these processes do not manage shared data properly, inconsistencies
  may arise. For instance, if two processes increment a shared counter
  without proper synchronization, the final value of the counter may be
  incorrect due to overlapping access. This issue is exacerbated in
  systems with multiple cores or processors, where different threads can
  execute simultaneously.
\item
  \textbf{Critical Sections and Mutual Exclusion:} A \textbf{critical
  section} is a part of a program that should only be executed by one
  process at a time to avoid data corruption. Ensuring \textbf{mutual
  exclusion}, where no two processes are in their critical sections
  simultaneously, is essential for preventing race conditions. Mutual
  exclusion can be achieved using synchronization mechanisms like
  \textbf{locks}, which ensure that only one process can access the
  critical section at any given time. Common techniques include
  \textbf{mutexes} and \textbf{binary semaphores}, both of which can
  lock shared resources to ensure orderly access. It is also important
  to minimize the time a process spends in a critical section to reduce
  the chances of blocking other processes.
\end{itemize}

\textbf{Deadlocks and Livelocks:}

\begin{itemize}
\item
  \textbf{Deadlocks} occur when processes are waiting on each other
  indefinitely, leading to a situation where no progress can be made.
  This typically happens in resource allocation scenarios where two or
  more processes form a circular chain of dependencies. A classic
  example is two processes holding a resource that the other needs to
  proceed, resulting in a stalemate.
\item
  \textbf{Livelocks} are similar to deadlocks but differ in that
  processes keep changing state in response to each other without making
  any actual progress. Both issues are critical in concurrent systems
  and need to be addressed through careful design, such as avoiding
  circular waiting conditions, implementing resource hierarchies, or
  using \textbf{deadlock detection algorithms} that can detect and break
  deadlocks. Timeout mechanisms can also be used to release resources if
  a process waits too long.
\end{itemize}

\section{Approaches to Manage
Concurrency}\label{approaches-to-manage-concurrency}

Concurrency in embedded systems is managed through a combination of
hardware and software synchronization mechanisms. Some of the methods
discussed include:

\textbf{Busy Waiting and Non-Busy Waiting:}

\begin{itemize}
\item
  \textbf{Busy waiting} occurs when a process remains in the ready
  state, actively checking for a condition to be satisfied. This
  approach is inefficient in terms of CPU utilization, as the CPU is
  occupied with checking instead of performing useful work.
\item
  \textbf{Non-busy waiting} solutions, such as putting a process in the
  blocked state while waiting for a resource, are more efficient for
  embedded systems where CPU resources are limited. Busy waiting is
  often used in systems where the wait time is expected to be very
  short, whereas non-busy waiting is preferred for longer wait times to
  free up CPU resources. In embedded systems, avoiding busy waiting is
  crucial because the CPU often needs to manage multiple tasks under
  strict timing constraints.
\item
  \textbf{Test and Set Instruction:} The \textbf{Test and Set}
  instruction is an atomic operation used to achieve mutual exclusion by
  using a global variable to indicate whether a critical section is
  occupied. By using Test and Set, embedded systems can ensure that
  processes access shared resources in a predictable and coordinated
  manner. The atomicity of this instruction prevents race conditions by
  ensuring that the check-and-set operation is performed without
  interruption. However, Test and Set can lead to \textbf{busy waiting},
  which may not be suitable for all embedded applications. In systems
  with limited processing power or battery life, busy waiting can
  degrade performance and reduce efficiency.
\end{itemize}

\textbf{Semaphores and Mutex Locks:}

\begin{itemize}
\item
  \textbf{Semaphores} are synchronization mechanisms used to manage
  concurrent access to shared resources. They allow processes to signal
  and wait for access, ensuring a controlled flow of execution.
  Semaphores can be \textbf{binary} (acting like a mutex) or
  \textbf{counting}, allowing a specified number of processes to access
  a resource concurrently. For example, counting semaphores are often
  used in producer-consumer scenarios where multiple producers or
  consumers need access to a shared buffer.
\item
  \textbf{Mutex locks} are similar but are typically used to protect
  access to a particular data structure or variable, providing mutual
  exclusion in critical sections. Mutexes are often simpler to implement
  but are limited to binary states (locked or unlocked), whereas
  semaphores offer more flexibility for resource management.
\end{itemize}

\textbf{Priority Inversion and Solutions:}

\begin{itemize}
\item
  \textbf{Priority inversion} occurs when a higher-priority task is
  waiting for a resource held by a lower-priority task, which can lead
  to missed deadlines in real-time systems. A well-known example is the
  \textbf{Mars Pathfinder mission}, where a low-priority task holding a
  shared resource caused a higher-priority task to miss its deadline,
  resulting in system resets.
\item
  \textbf{Solutions} to priority inversion include \textbf{priority
  inheritance}, where the lower-priority task temporarily inherits the
  higher priority to complete its task and release the resource. Another
  approach is \textbf{priority ceiling}, where a resource is assigned a
  priority equal to the highest priority of any task that may lock it,
  thus preventing priority inversion.
\end{itemize}

\section{Examples and Applications}\label{examples-and-applications}

The lecture provided various examples of where concurrency is critical,
including the automotive industry and robotics. For instance, the
\textbf{2010 Toyota recall} was attributed to software issues in their
anti-lock braking systems, highlighting the importance of deterministic
behavior in safety-critical systems. This incident shows how improperly
managed concurrency can lead to life-threatening failures. Similarly,
\textbf{Tesla's accidents} were referenced, emphasizing the challenges
of building reliable autonomous systems. These examples underscore the
need for proper concurrency management to ensure system reliability and
safety.

In embedded systems such as a \textbf{pacemaker}, concurrency is an
intrinsic characteristic. The pacemaker must concurrently monitor the
heart and trigger electrical pulses without delay, thus operating as a
real-time reactive system. Concurrency, in this context, is about
ensuring different components work seamlessly together to meet
life-critical deadlines. The pacemaker example illustrates how timing
constraints and concurrent interactions must be carefully managed to
avoid potentially fatal outcomes. The system must ensure that sensing,
decision-making, and actuation are carried out with precise timing to
maintain patient safety.

\section{Problems with Concurrency and
Solutions}\label{problems-with-concurrency-and-solutions}

Concurrency can lead to serious issues if not properly managed.

\textbf{Deadlocks}

\begin{itemize}
\tightlist
\item
  \textbf{Deadlocks} occur when concurrent processes are waiting on each
  other to release resources, resulting in a state where no process can
  proceed. To prevent deadlocks, proper scheduling mechanisms must be
  implemented to ensure that resource allocation does not lead to
  circular waiting. \textbf{Deadlock prevention techniques} include
  resource ordering, where resources are always requested in a specific
  order, and \textbf{deadlock detection algorithms} that periodically
  check for cycles in the resource allocation graph. Another approach is
  \textbf{deadlock avoidance}, such as using the \textbf{Banker's
  algorithm}, which allocates resources in a way that ensures a safe
  state is always maintained.
\end{itemize}

\textbf{Fairness and Scheduling}

\begin{itemize}
\item
  \textbf{Fairness} in resource allocation means that every process
  eventually gets a chance to execute. Embedded systems often use a
  \textbf{scheduler} to manage processes, ensuring tasks are completed
  in a timely and fair manner. The role of the scheduler is critical to
  prevent \textbf{starvation}, where some tasks never get executed due
  to others continuously consuming resources.
\item
  Common \textbf{scheduling} algorithms include \textbf{round-robin},
  \textbf{priority-based}, and \textbf{rate-monotonic scheduling}, each
  with its own advantages and trade-offs depending on the system
  requirements. \textbf{Round-robin scheduling} is simple and ensures
  fairness by giving each process a fixed time slice, while
  \textbf{priority-based scheduling} allows more critical tasks to
  execute first but can lead to starvation without priority aging.
\end{itemize}

\textbf{Starvation}

\begin{itemize}
\tightlist
\item
  \textbf{Starvation} occurs when a process is perpetually denied the
  resources it needs to execute, often because other higher-priority
  processes continuously take precedence. To mitigate starvation,
  \textbf{priority aging} can be used, where the priority of a waiting
  process is gradually increased over time to ensure it eventually gets
  the necessary resources. This technique ensures that lower-priority
  processes are not starved indefinitely, especially in systems with
  many competing processes.
\end{itemize}

\section{Conclusion}\label{conclusion}

Concurrency is an essential aspect of embedded systems, particularly for
reactive, real-time applications. This lecture emphasized the importance
of understanding the mechanisms---such as \textbf{mutual exclusion},
\textbf{semaphores}, and \textbf{scheduling}---that help manage
concurrent processes effectively. For embedded systems like automotive
controls, pacemakers, or industrial automation, \textbf{deterministic
behavior} and avoiding race conditions are paramount. Properly handling
concurrency ensures that embedded systems can meet their timing
constraints, maintain data consistency, and operate safely under all
conditions.

Next steps in the lecture will involve exploring \textbf{reactive
systems} and how \textbf{synchronous concurrency} can be implemented to
further ensure the reliability of embedded systems. Understanding these
concepts is vital for designing embedded systems that are safe,
reliable, and efficient. Future topics will also cover specific
synchronization protocols and \textbf{real-time operating system (RTOS)}
features that support concurrency in embedded environments.
Additionally, understanding how to leverage hardware features, such as
\textbf{interrupt controllers} and \textbf{timers}, will be crucial for
achieving precise control over concurrent tasks in embedded systems.

\chapter{SCCharts and Synchronous
Concurrency}\label{sccharts-and-synchronous-concurrency}

\section{Introduction to Synchronous Concurrency and
SCCharts}\label{introduction-to-synchronous-concurrency-and-sccharts}

Synchronous Concurrency (SC) is a method used to manage the complexity
of Cyber-Physical Systems (CPS) by providing deterministic behavior and
efficient handling of concurrent processes. SCCharts are a powerful
visual formalism used to model concurrent, reactive, and real-time
behaviors in embedded systems. They offer a way to design systems that
react to changes in the environment while maintaining reliable and
predictable execution.

\section{The Synchronous Approach}\label{the-synchronous-approach}

The \textbf{synchronous approach} is a paradigm in which all concurrent
processes are assumed to execute in lockstep relative to a global
logical clock. This provides a deterministic and predictable execution
model, which is highly beneficial for embedded systems that operate in
safety-critical environments.

Key aspects of the synchronous approach include:

\begin{itemize}
\item
  \textbf{Synchrony Hypothesis:} This hypothesis assumes that the
  system's reactions occur infinitely faster than the environment,
  leading to a \textbf{zero-delay model}. This means that the reaction
  to an input is instantaneous from the system's perspective, ensuring a
  deterministic response.
\item
  \textbf{Synchronous Threading and Broadcast Communication:} All
  concurrent threads execute synchronously, meaning that outputs
  generated by one component are immediately visible to all other
  components. This approach resembles \textbf{synchronous circuits},
  where each clock tick coordinates all activities, providing atomic and
  instantaneous reactions.
\item
  \textbf{Atomicity and Instantaneity of Reactions:} The synchronous
  approach treats reactions as atomic, meaning that no intermediate
  states are visible during execution. This helps prevent issues like
  race conditions, which are prevalent in asynchronous systems.
\item
  \textbf{Reaction to Absence:} The system can react not only to events
  that occur but also to the absence of expected events, allowing for a
  more comprehensive model of environmental interaction.
\end{itemize}

\section{SCCharts Overview}\label{sccharts-overview}

\textbf{SCCharts} are an extension of Statecharts, introduced by David
Harel, that add the benefits of the synchronous paradigm to the powerful
modeling features of Statecharts. SCCharts are used to describe complex
systems involving multiple states and concurrent processes.

\subsection{Similarities and Differences Between SCCharts and
Statecharts}\label{similarities-and-differences-between-sccharts-and-statecharts}

\begin{itemize}
\item
  \textbf{Similarities:} Both SCCharts and Harel's Statecharts share
  common elements such as states, transitions, signals/events, and
  modularity. They both support \textbf{hierarchy},
  \textbf{parallelism}, and \textbf{broadcast communication} between
  states, which allows for more compact and expressive models.
\item
  \textbf{Differences:} SCCharts operate within a \textbf{synchronous
  framework} and guarantee deterministic behavior, which contrasts with
  the often non-deterministic nature of Statecharts. SCCharts do not
  interpret events for simulations, which eliminates hidden behaviors
  and allows for more precise analysis. SCCharts also do not support
  \textbf{inter-level transitions} and provide deterministic handling of
  concurrent events, which simplifies debugging and validation.
\end{itemize}

\section{Boolean Mealy Machine (BMM)}\label{boolean-mealy-machine-bmm}

An SCChart can be represented using a \textbf{Boolean Mealy Machine
(BMM)}. A BMM is defined as a tuple
\(M = \langle Q, q_0, I, O, T \rangle\), where:

\begin{itemize}
\item
  \(Q\) is the set of states.
\item
  \(q_0\) is the initial state.
\item
  \(I\) is the set of inputs.
\item
  \(O\) is the set of outputs.
\item
  \(T\) is the transition relation, involving states and boolean
  conditions on inputs.
\end{itemize}

BMMs provide a formal basis for understanding how SCCharts operate, with
transitions being defined based on the current state and inputs. The key
properties of BMMs are \textbf{determinism} and \textbf{reactivity}:

\begin{itemize}
\item
  \textbf{Determinism}: At most one transition is enabled for any valid
  combination of inputs from the environment. This ensures that the
  system behaves predictably regardless of concurrent events.
\item
  \textbf{Reactivity}: At least one transition is enabled for any valid
  combination of inputs, ensuring the system can always respond
  appropriately to environmental changes.
\end{itemize}

\section{The VABRO Example}\label{the-vabro-example}

The \textbf{Valued ABRO (VABRO)} example, which is an adaptation of
Berry's \textbf{ABRO}, is used to demonstrate the synchronous
programming model. In this example, the system generates an output when
two specific inputs have occurred. The model incorporates \textbf{strong
pre-emption}, allowing the behavior to be reset when a particular
condition is met. VABRO demonstrates how SCCharts can handle complex
state-based logic, including counting events and generating output
values based on conditions.

\section{Constructiveness in Synchronous
Systems}\label{constructiveness-in-synchronous-systems}

\textbf{Constructiveness} is an important concept in synchronous
systems, ensuring that concurrent processes do not interfere with each
other's operation. \textbf{Berry's Constructiveness} requires that all
reads happen only after writes to shared entities are completed, and
that there is at most one write to any shared entity during a reaction.
This ensures consistency and prevents conflicting updates during a
single tick of the logical clock.

\textbf{Sequential Constructiveness} extends Berry's constructiveness by
allowing sequential updates to shared variables under certain
conditions. The \textbf{Init-Update-Read (IRU) protocol} is used to
manage variable updates in the following order:

\begin{enumerate}
\def\labelenumi{\arabic{enumi}.}
\item
  \textbf{Initialization}: Concurrent threads can make confluent
  initializations to shared variables, provided they assign the same
  value.
\item
  \textbf{Update}: Threads update shared variables using the same update
  function, ensuring confluent updates.
\item
  \textbf{Read}: Finally, threads are allowed to read the updated value,
  guaranteeing consistency.
\end{enumerate}

\section{Timed SCCharts and Real-Time
Systems}\label{timed-sccharts-and-real-time-systems}

\textbf{Timed SCCharts} add temporal constraints to SCCharts, enabling
the modeling of \textbf{real-time systems} where timing is crucial.
These models can compute the \textbf{Worst Case Reaction Time (WCRT)}
and adjust the tick length accordingly to ensure the system meets its
timing requirements. By counting ticks, timed SCCharts allow real-time
systems to be modeled effectively, ensuring that actions are taken
within the required time frame.

\textbf{Dynamic Ticks} are also introduced to allow the system to adjust
tick lengths dynamically, providing more flexibility in real-time
operation. This flexibility is essential for systems that need to adapt
to changing conditions, such as varying processing loads or external
inputs that influence timing requirements.

\section{Designing Real-Time Systems with
SCCharts}\label{designing-real-time-systems-with-sccharts}

The synchronous approach is particularly suitable for \textbf{real-time
systems}, where deterministic behavior is essential for safety and
reliability. By using SCCharts:

\begin{itemize}
\item
  \textbf{Concurrency is managed in a deterministic manner}, ensuring
  that all processes execute predictably without interference or
  unintended interactions.
\item
  \textbf{Pre-emption and Suspension} are modeled explicitly, allowing
  designers to represent high-priority tasks interrupting lower-priority
  tasks, or suspending ongoing activities until a condition is met.
\item
  The \textbf{generated code from SCCharts} compiles the concurrent
  specifications into sequential code, which is efficient and suitable
  for \textbf{bare metal implementations} without requiring an operating
  system. This makes SCCharts an excellent choice for systems with
  constrained resources, such as microcontrollers in embedded
  environments.
\end{itemize}

\section{Conclusion}\label{conclusion-1}

Synchronous concurrency, as implemented using SCCharts, provides a
robust framework for designing \textbf{deterministic, reactive, and
real-time embedded systems}. By combining the power of Statecharts with
the guarantees of the synchronous model, SCCharts offer a visual and
formal way to specify complex concurrent behavior while ensuring
predictable execution. The \textbf{synchronous approach} is well-suited
for \textbf{safety-critical CPS} applications, such as medical devices,
automotive systems, and robotics, where the cost of failure is high.

Future discussions will delve deeper into \textbf{synchronous
pre-emption}, \textbf{timed automata}, and how SCCharts can be extended
to model more complex temporal behaviors. These tools and concepts are
crucial for ensuring that CPS can meet their real-time requirements in a
safe and reliable manner.

\part{Avinash's Lectures}

\chapter{Peripherals and Interfacing}\label{peripherals-and-interfacing}

\section{Introduction to Peripherals and Interfacing in Embedded
Systems}\label{introduction-to-peripherals-and-interfacing-in-embedded-systems}

Embedded systems require effective mechanisms for communicating with
various external devices, known as \textbf{peripherals}. These
peripherals, which include sensors, actuators, timers, and communication
modules, interact with the processor through specialized interfaces
(Slide 4). \textbf{Interfacing} is critical in embedded systems as it
determines how well the processor can manage peripherals to meet
performance and reliability requirements. The lecture discussed various
types of I/O, including memory-mapped and direct I/O, as well as data
transfer methods like polling, interrupts, and Direct Memory Access
(DMA).

\section{Types of I/O Interfaces}\label{types-of-io-interfaces}

\begin{itemize}
\item
  \textbf{Memory-Mapped I/O}: In this type of I/O, the processor treats
  peripheral registers as memory locations within the same address space
  (Slide 7). The advantage of this approach is that the same
  instructions can be used for both memory and I/O operations,
  simplifying program design. However, memory-mapped I/O also shares
  address lines with memory, which means it can impact addressable
  memory space.

  \begin{itemize}
  \tightlist
  \item
    Example: The \textbf{Nios II processor} from Altera uses
    memory-mapped I/O to communicate with peripherals via Avalon buses,
    where specific addresses map to different peripheral registers, such
    as control, status, or data registers (Slide 7). Functions like
    \texttt{IORD()} and \texttt{IOWR()} are used to read from and write
    to these registers.
  \end{itemize}
\item
  \textbf{Direct I/O}: Here, peripherals are addressed using separate
  I/O ports, distinct from the memory address space (Slide 8). This
  method requires specialized I/O instructions, such as \texttt{IN} and
  \texttt{OUT} on x86 processors, making it more suitable for systems
  where efficient separation of I/O and memory space is required. In
  \textbf{direct I/O}, the processor has separate control signals
  specifically for enabling devices, which prevents interference with
  memory operations.
\end{itemize}

\section{Data Transfer Methods}\label{data-transfer-methods}

\begin{itemize}
\item
  \textbf{Programmed I/O}: In this mode, the processor executes a
  program to directly manage data transfer between itself and the I/O
  device (Slide 11). While simple to implement, it is inefficient since
  the processor is blocked during the entire transfer process, often
  resulting in idle CPU cycles while waiting for the peripheral to be
  ready.
\item
  \textbf{Interrupt-Driven I/O}: This method leverages
  \textbf{interrupts} to initiate data transfer, which improves
  efficiency by allowing the processor to perform other tasks until the
  I/O device signals readiness (Slide 14). When data is ready, the
  peripheral asserts an interrupt signal, causing the processor to
  execute an \textbf{Interrupt Service Routine (ISR)}. The ISR then
  handles the data transfer, enabling better utilization of CPU
  resources compared to polling.

  \begin{itemize}
  \tightlist
  \item
    \textbf{Fixed and Vectored Interrupts}: With \textbf{fixed
    interrupts}, the address of the ISR is built into the
    microprocessor, while \textbf{vectored interrupts} allow the
    peripheral to specify the address of its ISR, providing more
    flexibility (Slide 16-24). Interrupts improve system efficiency but
    add complexity due to the need for context saving and careful ISR
    design.
  \end{itemize}
\item
  \textbf{DMA (Direct Memory Access)}: DMA involves an external
  controller taking over the system bus to manage data transfer between
  a peripheral and memory without processor involvement, freeing up the
  CPU for other operations (Slide 37). \textbf{Block DMA} transfers data
  in large blocks, while \textbf{cycle-stealing DMA} transfers a few
  bytes at a time, relinquishing the bus to the CPU when needed (Slide
  52).

  \begin{itemize}
  \tightlist
  \item
    \textbf{Example}: In a \textbf{peripheral-to-memory transfer using
    DMA}, the DMA controller initiates a request to gain control of the
    bus, then proceeds to transfer data while the CPU resumes its normal
    execution. The CPU only stalls when it requires access to the bus,
    significantly improving system performance by overlapping
    computation and data transfer (Slide 46-50).
  \end{itemize}
\end{itemize}

\section{Timer and Counter
Peripherals}\label{timer-and-counter-peripherals}

Timers and counters are integral parts of embedded systems for handling
periodic events and measuring intervals (Slide 53).

\begin{itemize}
\item
  \textbf{Timers}: Timers are used to measure time intervals or generate
  timed output events. For example, a \textbf{16-bit timer} counting
  clock pulses with a 10 ns period can generate an event after a set
  duration. Timers can be used in applications like controlling traffic
  lights or measuring a car's speed.
\item
  \textbf{Counters}: Similar to timers but used for counting external
  pulses, such as counting cars passing over a sensor.
  \textbf{Programmable timers} allow setting a terminal count value to
  specify when an event should occur, which is critical for tasks like
  interval timing or watchdog functionalities (Slide 54).
\item
  \textbf{Example of Reaction Timer}: A \textbf{reaction timer} measures
  the user's response time to a visual stimulus (Slide 56). A timer is
  started when an indicator light turns on, and the user's reaction time
  is recorded by counting the number of timer overflows until a button
  is pressed. This showcases the use of timers to measure user
  interaction intervals.
\end{itemize}

\section{UART Communication}\label{uart-communication}

\textbf{Universal Asynchronous Receiver/Transmitter (UART)} is a widely
used peripheral for serial communication in embedded systems. It
converts parallel data from the processor to serial data for
transmission and vice versa (Slide 58).

\begin{itemize}
\item
  \textbf{Data Transmission and Reception}: UART communication uses
  start, data, and stop bits to frame each byte of information, with an
  optional parity bit for error detection. UARTs require synchronization
  mechanisms like \textbf{baud rate} settings to ensure the transmitter
  and receiver operate at the same speed (Slide 58).
\item
  \textbf{Registers and Flow Control}: UARTs often use multiple
  registers, such as \textbf{control}, \textbf{status}, \textbf{rxdata},
  and \textbf{txdata} (Slide 60). For example, a transmitter may set the
  \textbf{Request to Send (RTS)} bit in the control register and then
  poll the \textbf{Clear to Send (CTS)} bit to determine if the receiver
  is ready. Double buffering in the transmitter and receiver allows for
  more efficient communication by enabling one data transfer while
  another is in progress.
\item
  \textbf{Interrupt Handling in UARTs}: The UART peripheral can generate
  interrupts when new data is received or the transmission buffer is
  ready for new data, thus enabling efficient communication without
  busy-waiting (Slide 64). The ISR reads the \textbf{rxdata} register
  and processes the incoming data, which is crucial in scenarios where
  timely responses are needed, such as in real-time data logging.
\end{itemize}

\section{Peripheral Interfacing
Techniques}\label{peripheral-interfacing-techniques}

\textbf{Interrupt Handling in Nios II Processors}: The \textbf{Nios II
processor} uses an exception handler provided by the Hardware
Abstraction Layer (HAL) to differentiate between software exceptions and
hardware interrupts (Slide 31). This approach ensures efficient response
handling based on the nature of the interrupt. A typical function for
registering an interrupt in Nios II is \texttt{alt\_irq\_register()},
where the handler function and device ID are specified (Slide 32).

\textbf{Direct Memory Access for High-Speed Transfers}: DMA is employed
when dealing with high-speed peripherals that need to transfer large
data blocks. For example, transferring data from a tape drive or video
feed requires a DMA controller to handle the high volume of data without
overloading the CPU (Slide 37). The use of \textbf{DMA controllers}
helps avoid bottlenecks in data transfer and allows parallel processing,
which is essential in high-performance embedded systems.

\section{Practical Considerations and
Limitations}\label{practical-considerations-and-limitations}

\textbf{Limitations of Programmed I/O}: Programmed I/O is unsuitable for
high-speed peripherals because it involves significant overhead, where
the CPU needs to execute multiple instructions for each word of data
transferred (Slide 36). This leads to inefficiencies, particularly when
handling peripherals that require large amounts of data to be processed
rapidly, such as audio or video streams.

\textbf{DMA vs Programmed I/O}: The main advantage of \textbf{DMA} over
programmed I/O is its ability to handle large data transfers without
constant CPU intervention, reducing the overhead on the main processor
(Slide 38). This is particularly useful for applications like
transferring video frames to memory or storing data from high-speed
sensors.

\section{Conclusion}\label{conclusion-2}

\textbf{Peripherals and interfacing} are fundamental components of
embedded systems, enabling interaction with the external world. The
lecture covered various data transfer methods, highlighting their
advantages and trade-offs. Effective use of \textbf{interrupts},
\textbf{DMA}, and \textbf{timers} allows embedded systems to operate
efficiently, especially in real-time applications where timely responses
are critical. Understanding the different modes of I/O---such as
memory-mapped versus direct I/O---and the specific requirements for
handling data transfers through \textbf{polling}, \textbf{interrupts},
or \textbf{DMA} helps in designing robust and responsive embedded
systems.

Future discussions will focus on specific interfacing techniques for
more complex peripherals, such as \textbf{SPI} and \textbf{I2C}
communication protocols, as well as how to optimize
\textbf{interrupt-driven I/O} to minimize latency and maximize
throughput in resource-constrained environments.

\chapter{Embedded Processors}\label{embedded-processors}

\section{Introduction to Embedded
Processors}\label{introduction-to-embedded-processors}

Embedded processors are specialized computation engines designed to
implement a system's desired functionality. Unlike general-purpose
processors, embedded processors can be optimized for specific
applications, balancing performance, power consumption, and area
constraints (Slide 5). There are three main types of embedded
processors: \textbf{general-purpose processors}, \textbf{single-purpose
processors}, and \textbf{application-specific processors}. Each has its
advantages and trade-offs, depending on the intended application.

\section{Types of Embedded
Processors}\label{types-of-embedded-processors}

\subsection{General-Purpose
Processors}\label{general-purpose-processors}

\textbf{General-purpose processors} (also known as microprocessors) are
programmable devices used in a wide variety of applications. They
contain a \textbf{program memory}, a \textbf{datapath} with a large
register file, and a \textbf{general ALU} (Slide 7). Some notable
features include:

\begin{itemize}
\item
  \textbf{High Flexibility}: General-purpose processors can run
  different software programs, making them highly adaptable.
\item
  \textbf{Low Non-Recurring Engineering (NRE) Costs}: These processors
  can be developed for a broad market, leading to lower initial costs.
\item
  \textbf{Examples}: The \textbf{Pentium} processor is a well-known
  example, but there are many other general-purpose processors used in
  embedded systems.
\end{itemize}

General-purpose processors are ideal for systems that need versatility
but do not have extreme performance or power constraints.

\subsection{Single-Purpose Processors}\label{single-purpose-processors}

\textbf{Single-purpose processors} are digital circuits designed to
execute only one specific program, such as a \textbf{coprocessor},
\textbf{accelerator}, or \textbf{peripheral} (Slide 8). Key
characteristics include:

\begin{itemize}
\item
  \textbf{Minimal Components}: They contain only the elements necessary
  for their task, often without program memory, which reduces size and
  power consumption.
\item
  \textbf{Benefits}: Single-purpose processors are \textbf{fast},
  \textbf{small}, and \textbf{low power} due to their dedicated nature.
  Since there are no extra components, they achieve high efficiency in
  both speed and energy.
\item
  \textbf{Examples}: Hardware implementations of functions like
  \textbf{Greatest Common Divisor (GCD)} are often realized with
  single-purpose processors.
\end{itemize}

These processors are well-suited for real-time, deterministic tasks that
require minimal latency and maximum energy efficiency.

\subsection{Application-Specific
Processors}\label{application-specific-processors}

\textbf{Application-specific processors} are programmable processors
optimized for a particular class of applications that share common
characteristics (Slide 9). They combine aspects of both general-purpose
and single-purpose processors:

\begin{itemize}
\item
  \textbf{Program Memory and Optimized Datapath}: Application-specific
  processors include a program memory and a \textbf{customized datapath}
  tailored to the target applications, such as \textbf{DSPs} (Digital
  Signal Processors).
\item
  \textbf{Trade-Off}: They offer some flexibility compared to
  single-purpose processors, while providing better performance and
  efficiency than general-purpose processors.
\item
  \textbf{Benefits}: These processors provide a balance of flexibility,
  performance, size, and power consumption, making them ideal for use in
  systems that need customization without sacrificing too much
  generalizability.
\end{itemize}

\section{Finite-State Machine with Datapath
(FSMD)}\label{finite-state-machine-with-datapath-fsmd}

One of the most common approaches in custom embedded processor design is
the use of a \textbf{Finite-State Machine with Datapath (FSMD)}. FSMD is
a model that integrates control and data paths, allowing complex
behaviors to be implemented systematically (Slide 17).

\begin{itemize}
\item
  \textbf{FSMD Structure}: An FSMD consists of:

  \begin{itemize}
  \item
    \textbf{Controller}: A finite-state machine that determines the
    sequence of operations.
  \item
    \textbf{Datapath}: Composed of \textbf{registers}, \textbf{ALUs},
    and \textbf{other functional units} used to manipulate data (Slide
    19).
  \end{itemize}
\item
  \textbf{Design Flow}: Typically, a design starts by creating an
  algorithm, converting it to a state machine, and then mapping it to an
  FSMD (Slide 18). This involves defining states, transitions, and
  operations that control how data is processed.
\item
  \textbf{Example: GCD Calculation}: The \textbf{Greatest Common Divisor
  (GCD)} example demonstrated in the lecture involves defining a simple
  algorithm and converting it to an FSMD. The states of the GCD FSM
  correspond to different stages of the computation, and the datapath
  manages the subtraction operations required to compute the GCD (Slide
  17).
\end{itemize}

\section{Custom Processor Design
Techniques}\label{custom-processor-design-techniques}

The design of custom embedded processors involves several key techniques
for optimizing performance, size, and power:

\begin{itemize}
\item
  \textbf{Datapath Design}: Involves creating registers for each
  declared variable and functional units for each arithmetic operation
  (Slide 19). The registers, ALU, and functional units are connected
  based on the data flow in the original algorithm.
\item
  \textbf{Controller Design}: The controller is responsible for
  sequencing the operations in the datapath. Complex actions in the FSMD
  are broken down into manageable control steps to ensure deterministic
  behavior (Slide 20).
\item
  \textbf{Splitting into Controller and Datapath}: Splitting the
  controller and datapath allows for better modularity and optimization
  of the design. The controller handles decision-making, while the
  datapath focuses on arithmetic and logical operations (Slide 21).
\end{itemize}

\section{Optimization Strategies for Embedded
Processors}\label{optimization-strategies-for-embedded-processors}

\textbf{Optimization} is the process of refining the design to achieve
the best possible trade-off among different design metrics, such as
speed, area, and power consumption (Slide 24).

\begin{itemize}
\item
  \textbf{Program Optimization}: Involves analyzing the original program
  for inefficiencies. The use of more efficient arithmetic operations or
  changing the sequence of operations can significantly improve the
  overall performance. For example, replacing subtraction operations
  with modulo in the GCD algorithm sped up the calculation and reduced
  loop iterations (Slide 26).
\item
  \textbf{FSMD Optimization}: The FSMD can be optimized by merging or
  separating states to improve efficiency (Slide 27). \textbf{State
  merging} can be done for states with constant transitions, whereas
  \textbf{state separation} can help reduce hardware complexity for
  states requiring complex operations.
\item
  \textbf{Datapath Optimization}: \textbf{Sharing functional units}
  between states can reduce the area by avoiding redundant hardware.
  \textbf{Multi-functional units} like ALUs can be used for different
  operations in different states, optimizing resource usage (Slide 29).
\item
  \textbf{FSM Optimization}: Optimizing the finite-state machine
  involves techniques like \textbf{state encoding} (assigning efficient
  binary codes to states) and \textbf{state minimization} (merging
  equivalent states to reduce complexity) (Slide 30).
\end{itemize}

\section{Custom Single-Purpose Processor Design: An
Example}\label{custom-single-purpose-processor-design-an-example}

The lecture provided an example of designing a custom single-purpose
processor for a simple \textbf{bridge} that converts two 4-bit inputs
into an 8-bit output (Slide 31).

\begin{itemize}
\item
  \textbf{RT-Level Design}: The Register-Transfer (RT) level model is
  used to specify how data is transferred and manipulated in each clock
  cycle. The design includes both a \textbf{controller FSM} and a
  \textbf{datapath} with registers and functional units (Slide 32).
\item
  \textbf{State Diagram and Data Flow}: The RT-level processor's
  behavior is defined through state diagrams and data flow diagrams,
  demonstrating how each input is processed and combined into an output.
  The controller is responsible for sequencing these operations, while
  the datapath handles the actual data manipulation.
\end{itemize}

\section{C to Hardware: Synthesis and
Implementation}\label{c-to-hardware-synthesis-and-implementation}

Another important aspect covered in the lecture was the conversion from
\textbf{C code to hardware}. This is known as \textbf{High-Level
Synthesis (HLS)} and is used to automatically convert behavioral C code
into corresponding hardware designs (Slide 14).

\begin{itemize}
\item
  \textbf{Motivation}: The main advantage is achieving maximum
  performance for specific applications without having to manually write
  hardware descriptions in HDL (Hardware Description Language). This
  approach allows software developers to leverage existing C code and
  translate it into efficient hardware implementations.
\item
  \textbf{Components}: The translation involves identifying data and
  control instructions and converting them into datapath elements like
  \textbf{ALUs}, \textbf{registers}, and \textbf{control logic}. This is
  then used to create a hardware implementation of the algorithm
  specified in C (Slide 15).
\end{itemize}

\section{Evolution of Embedded
Processors}\label{evolution-of-embedded-processors}

Embedded processors have evolved significantly to meet the demands of
modern applications (Slide 10).

\begin{itemize}
\item
  \textbf{Single-Purpose to Application-Specific Processors}: Initially,
  systems relied on \textbf{single-purpose processors} for dedicated
  tasks. These processors provided efficient solutions but lacked
  flexibility.
\item
  \textbf{Application-Specific Processors}: As the complexity of
  embedded applications increased, the need for processors that strike a
  balance between performance and flexibility led to the development of
  \textbf{application-specific processors}. These processors retain the
  programmability of general-purpose processors while including
  optimizations for particular tasks.
\item
  \textbf{Customizable Processors}: Modern approaches include the use of
  \textbf{customizable processors} that allow designers to optimize
  hardware for a specific application domain without losing the general
  flexibility of programmable processors.
\end{itemize}

\section{Conclusion}\label{conclusion-3}

The lecture provided a comprehensive overview of \textbf{embedded
processors} and their role in implementing complex systems. Custom
embedded processors, including general-purpose, single-purpose, and
application-specific types, each have their unique trade-offs.
Techniques such as \textbf{FSMD modeling}, \textbf{controller and
datapath splitting}, and \textbf{high-level synthesis} were introduced
to demonstrate the design and optimization of these processors.

Future discussions will delve deeper into \textbf{real-time
constraints}, \textbf{hardware-software co-design}, and techniques for
optimizing \textbf{multi-core embedded systems} to meet stringent
performance and power requirements. Understanding these concepts is
essential for designing embedded processors that can efficiently address
the challenges posed by modern applications.

\chapter{SoC and SOPC Buses}\label{soc-and-sopc-buses}

\section{Introduction to SoC and
SOPC}\label{introduction-to-soc-and-sopc}

\textbf{System-on-Chip (SoC)} and \textbf{System-on-Programmable-Chip
(SOPC)} are key technologies in modern embedded systems. These
technologies combine multiple functional blocks, such as processors,
memory, and peripherals, onto a single chip. They use \textbf{buses} to
facilitate communication between these components (Slide 1). Buses are
fundamental to the functionality of SoC and SOPC designs, as they
determine how different modules interact, ensuring reliable data
transfer and system performance.

\section{Internal vs External Buses}\label{internal-vs-external-buses}

Buses can be broadly classified into \textbf{internal} and
\textbf{external} types based on their scope within a system (Slide 9):

\begin{itemize}
\item
  \textbf{Internal Buses}: These are used within a single chip or a
  tightly integrated board to connect different functional blocks.
  Examples include:

  \begin{itemize}
  \item
    \textbf{PCI} (Peripheral Component Interconnect)
  \item
    \textbf{AGP} (Accelerated Graphics Port)
  \item
    \textbf{PCMCIA}
  \item
    \textbf{AMBA} (Advanced Microcontroller Bus Architecture) for
    ARM-based systems
  \item
    \textbf{Avalon} buses used in Altera's SOPC designs
  \end{itemize}
\item
  \textbf{External Buses}: These are used to connect different chips or
  boards. Examples include:

  \begin{itemize}
  \item
    \textbf{USB} (Universal Serial Bus)
  \item
    \textbf{FireWire}
  \end{itemize}
\end{itemize}

\section{Bus Design Issues}\label{bus-design-issues}

The design of a bus impacts system performance, scalability, and cost.
Several factors must be considered when designing a bus (Slide 14):

\begin{itemize}
\item
  \textbf{Bus Width}: The width of the data and address buses is a
  critical determinant of the system's performance. A wider bus can
  transfer more data per cycle but is more expensive to implement. For
  example, a Pentium processor has a 32-bit instruction set architecture
  (ISA) but a 64-bit data bus, while an Itanium processor has a 128-bit
  data bus, offering significantly more bandwidth for data transfers
  (Slides 15-16).
\item
  \textbf{Bus Type}: Buses can be \textbf{dedicated} or
  \textbf{multiplexed}:

  \begin{itemize}
  \item
    \textbf{Dedicated Buses} have separate lines for data and address
    information, providing better performance but increasing costs.
  \item
    \textbf{Multiplexed Buses} share the same lines for data and
    address, which reduces cost and resource usage but may introduce
    delays in data transmission (Slide 17).
  \end{itemize}
\end{itemize}

\section{Bus Operations}\label{bus-operations}

\textbf{Bus operations} define how data is transferred between
components (Slide 18). Common bus operations include:

\begin{itemize}
\item
  \textbf{Read and Write}: Basic data transfer operations.
\item
  \textbf{Block Transfer}: A series of contiguous memory locations are
  read or written in sequence, useful for operations like filling a
  cache line.
\item
  \textbf{Read-Modify-Write}: Often used for critical sections where
  multiple operations must be performed atomically.
\item
  \textbf{Interrupt Operation}: A mechanism for peripherals to request
  the processor's attention, triggering an interrupt service routine.
\end{itemize}

\section{Synchronous vs Asynchronous
Buses}\label{synchronous-vs-asynchronous-buses}

\subsection{Synchronous Bus}\label{synchronous-bus}

A \textbf{synchronous bus} uses a \textbf{bus clock signal} to
synchronize all actions on the bus (Slide 19). Characteristics of
synchronous buses include:

\begin{itemize}
\item
  \textbf{Clock Edge Coordination}: All data transfers occur relative to
  a clock signal, simplifying timing design.
\item
  \textbf{Wait States}: A synchronous bus may operate with \textbf{wait
  states} if the target device is not ready for a data transfer. Without
  wait states, data transfers occur at the maximum clock speed, but
  adding wait states ensures reliable operation if the target is slower
  (Slides 20-24).
\item
  \textbf{Block Transfers}: Synchronous buses also support \textbf{block
  transfers}, which allow multiple data units to be transferred in a
  single bus transaction, improving throughput (Slide 24).
\end{itemize}

\subsection{Asynchronous Bus}\label{asynchronous-bus}

An \textbf{asynchronous bus} does not use a clock signal to synchronize
actions. Instead, it relies on \textbf{handshaking signals} to control
data transfer between the master and slave (Slide 25). Features include:

\begin{itemize}
\item
  \textbf{Handshaking Protocol}: Two synchronization signals,
  \textbf{Master Synchronization (MSYN)} and \textbf{Slave
  Synchronization (SSYN)}, ensure that the sender and receiver are ready
  for each data transfer.
\item
  \textbf{No Clock Dependency}: This allows asynchronous buses to be
  more flexible, as they do not require a fixed clock frequency.
  However, implementing the handshaking mechanism can be more complex
  compared to synchronous buses.
\end{itemize}

\section{Bus Arbitration}\label{bus-arbitration}

In systems with multiple masters (e.g., CPUs, DMA controllers),
\textbf{bus arbitration} is required to manage access to the bus (Slide
27). Arbitration mechanisms include:

\begin{itemize}
\item
  \textbf{Static Arbitration}: Allocates the bus in a fixed,
  predetermined manner. While easy to implement, this approach can lead
  to poor utilization and potential bus monopolization by one master
  (Slide 28).
\item
  \textbf{Dynamic Bus Arbitration}: Allocates the bus only when a master
  requests it, allowing more efficient use of resources. Masters use
  \textbf{bus request} and \textbf{bus grant} lines to signal their need
  for the bus and receive permission to use it. Dynamic arbitration
  supports various allocation policies (Slides 28-40):

  \begin{itemize}
  \item
    \textbf{Fixed Priority}: Assigns a fixed priority to each master,
    but may lead to \textbf{bus hogging} by high-priority devices.
  \item
    \textbf{Rotating Priority}: Changes the priority of masters
    dynamically to prevent starvation. The \textbf{lowest priority} is
    assigned to the master that just used the bus, ensuring fair access.
  \item
    \textbf{Fair Policies}: Ensure that no master is starved of the bus.
    These policies can be based on \textbf{time windows} to guarantee a
    maximum delay before a bus request is granted.
  \end{itemize}
\item
  \textbf{Bus Release Policies}: Determine when the current master
  releases the bus. Two types of policies are:

  \begin{itemize}
  \item
    \textbf{Non-Preemptive Release}: The master voluntarily releases the
    bus after completing the current transaction. This is easy to
    implement but can lead to inefficient bus usage if the master holds
    the bus unnecessarily (Slide 40).
  \item
    \textbf{Preemptive Release}: Forces the current master to release
    the bus, which can be useful for handling urgent requests from other
    masters.
  \end{itemize}
\end{itemize}

\subsection{Bus Arbitration
Implementation}\label{bus-arbitration-implementation}

\textbf{Centralized Bus Arbitration}: Uses a \textbf{daisy-chaining}
method where the bus grant signal is passed down a chain of masters. It
is easy to implement but has several drawbacks, such as fixed priority
and increased arbitration time proportional to the number of masters
(Slide 37).

\textbf{Independent Requests}: Each master has a dedicated bus request
line to the central arbiter. This approach is more fault-tolerant and
allows for \textbf{constant arbitration time}, but requires a higher
number of control signals (Slide 39).

\textbf{Hybrid Arbitration Schemes}: Combine features of daisy-chaining
and independent requests. For example, bus masters can be grouped into
\textbf{classes}, with independent arbitration at the class level and
daisy-chaining within each class. This provides a balance between cost
and performance (Slide 40).

\section{System Bus Components}\label{system-bus-components}

The system bus consists of the following components:

\begin{itemize}
\item
  \textbf{Address Bus}: Carries the addresses from the master to the
  slave, specifying the location for read or write operations.
\item
  \textbf{Data Bus}: Transfers actual data between components.
\item
  \textbf{Control Bus}: Carries control signals, such as read/write
  enable, to coordinate the operations on the address and data buses
  (Slide 12).
\end{itemize}

Buses can be either \textbf{dedicated} or \textbf{multiplexed},
\textbf{synchronous} or \textbf{asynchronous}, depending on the system's
requirements for speed, cost, and complexity (Slide 12).

\section{Learning Outcomes}\label{learning-outcomes}

By understanding the different types of buses, operations, arbitration
schemes, and bus components, one can design effective \textbf{SoC and
SOPC} systems that balance performance, cost, and complexity. Key
takeaways include:

\begin{itemize}
\item
  The importance of selecting appropriate \textbf{bus width} to match
  system requirements while considering cost implications.
\item
  Differences between \textbf{synchronous} and \textbf{asynchronous}
  buses and the trade-offs involved in choosing one over the other.
\item
  The necessity of \textbf{bus arbitration} in multi-master systems to
  ensure fair access and system efficiency.
\item
  How different arbitration policies impact bus access, latency, and
  fairness.
\end{itemize}

The \textbf{building blocks of an SoC or SOPC}---including
\textbf{masters (e.g., processors, DMA controllers)}, \textbf{slaves
(e.g., peripherals)}, \textbf{buses}, \textbf{decoders}, and
\textbf{arbiters}---all work together to create a cohesive, functional
system capable of handling complex tasks (Slide 53).

\section{Conclusion}\label{conclusion-4}

The lecture provided a comprehensive overview of \textbf{bus design and
operations} in SoC and SOPC systems. Understanding bus types,
operations, arbitration, and implementation is crucial for designing
efficient and scalable embedded systems. These principles are applied to
ensure that various components within a SoC or SOPC can communicate
effectively while meeting performance requirements.

Future discussions will explore \textbf{on-chip interconnects},
\textbf{crossbars}, and \textbf{network-on-chip (NoC)} architectures,
which are becoming increasingly relevant for \textbf{multi-core SoC
designs} that demand higher data throughput and efficient parallel
processing.

\part{Nathan's Lectures}

\chapter{Embedded Control Systems 1}\label{embedded-control-systems-1}

\section{Introduction to Embedded Control
Systems}\label{introduction-to-embedded-control-systems}

\textbf{Embedded control systems} are crucial for industrial automation
and other real-time applications that require precise, timely responses.
These systems often control physical processes, ensuring safety and
reliability through careful computational analysis of execution times.
This lecture focused on various elements of embedded control, such as
real-time system properties, sampling and quantization, computational
delays, and worst-case execution time (WCET) analysis (Slides 2-3).

\section{Real-Time Systems}\label{real-time-systems}

\textbf{Definition}: Real-time systems are used in scenarios where
timing is critical, such as industrial automation and safety-critical
applications. These systems must perform their functions not only
correctly (functional correctness) but also within specific timing
constraints (Slide 3).

\textbf{Examples}: A robotic arm must stop within 10 ms of detecting
human intrusion to prevent accidents. This combination of functional and
timing correctness is what defines a real-time system.

\section{Sampling and Quantization}\label{sampling-and-quantization}

\textbf{Sampling}: Sampling refers to taking discrete snapshots of an
analog signal over time. This process is fundamental in digitizing
signals like video, audio, or physical voltages. The higher the sampling
rate, the more accurately the original signal is represented (Slides 4,
8).

\textbf{Quantization}: Due to the discrete nature of computer data, real
values must be approximated to the nearest value in a finite set.
Quantization replaces real-world continuous values with discrete levels,
introducing \textbf{quantization error} (Slide 5).

\begin{itemize}
\item
  \textbf{Quantization Levels}: For an n-bit Analog-to-Digital Converter
  (ADC), there are \(2^n\) levels, which determine how finely the signal
  is represented. For example, an 8-bit ADC has 256 quantization levels.
\item
  \textbf{Resolution}: The resolution of an ADC determines the voltage
  represented by each digital increment. For instance, a 10-bit ADC with
  a High Reference Voltage (HRV) of 5V and Low Reference Voltage (LRV)
  of 0V has a resolution of approximately 0.00488V per step (Slide 6).
\end{itemize}

\section{Aliasing}\label{aliasing}

\textbf{Aliasing} occurs due to insufficient sampling rates. If the
sampling rate is less than twice the frequency of the signal (the
\textbf{Nyquist frequency}), higher-frequency signals appear
indistinguishable from lower frequencies, leading to errors (Slides
9-11).

\begin{itemize}
\tightlist
\item
  For example, if a signal is sampled at 2.5 Hz, signals at 0.5 Hz, 3
  Hz, and 5.5 Hz appear the same, leading to ambiguity.
\end{itemize}

\section{Computation Delays and Timing
Variances}\label{computation-delays-and-timing-variances}

\textbf{Computation Delays}: In real-time systems, delays can occur due
to inherent processing overhead, causing sampling or actuation to happen
later than expected. Accurate delay characterization is needed to ensure
that systems meet timing requirements (Slide 12).

\begin{itemize}
\item
  \textbf{Sources of Timing Variance}:

  \begin{itemize}
  \item
    \textbf{Unbounded Software Constructs}: Unpredictable branches,
    loops, recursion, and dynamic memory allocations can introduce
    delays (Slide 13).
  \item
    \textbf{Speculative Hardware}: Features like pipelines, memory
    hierarchies (caches), and branch predictors in modern processors can
    lead to non-deterministic execution times.
  \end{itemize}
\end{itemize}

\section{Worst-Case Execution Time
(WCET)}\label{worst-case-execution-time-wcet}

\textbf{WCET} is the maximum time required for a system to complete a
given task (Slide 14). Understanding WCET is critical for ensuring that
an embedded control system meets its real-time constraints.

\begin{itemize}
\item
  \textbf{Methods to Determine WCET}:

  \begin{itemize}
  \item
    \textbf{Measured}: Simple to implement but may not explore all
    possible execution paths, thus providing non-guaranteed values.
  \item
    \textbf{Static Analysis}: Uses an abstraction of the system to
    analyze all possible paths, ensuring coverage but requiring
    significant computational resources.
  \end{itemize}
\end{itemize}

\section{Static Timing Analysis}\label{static-timing-analysis}

\begin{itemize}
\item
  \textbf{Control Flow Graph (CFG)}: Timing analysis starts with
  creating a \textbf{control flow graph} (CFG) to map program flow,
  representing different segments of code (Slides 18-19).

  \begin{itemize}
  \item
    \textbf{Basic Blocks and Transitions}: CFG nodes are called
    \textbf{basic blocks}, which represent straight-line code without
    branches, and transitions show how the program moves between these
    blocks.
  \item
    \textbf{Path Analysis}: Finding the \textbf{longest execution path}
    within the CFG is key to determining the worst-case scenario for
    execution time (Slide 20).
  \end{itemize}
\item
  \textbf{Timed CFG}: A timed CFG uses clock cycle estimates for
  different operations---computation, load, store, branch, and
  return---to determine execution duration (Slide 23).

  \begin{itemize}
  \tightlist
  \item
    Example values: \textbf{Load} and \textbf{Store} take 5 cycles,
    \textbf{Computation} takes 1 cycle, \textbf{Branch} (if taken) takes
    3 cycles.
  \end{itemize}
\end{itemize}

\section{WCET Analysis Using Max-Plus
Algebra}\label{wcet-analysis-using-max-plus-algebra}

\begin{itemize}
\item
  The \textbf{Max-Plus} approach simplifies WCET analysis by using only
  max and plus operations to calculate the worst-case path cost (Slide
  24).

  \begin{itemize}
  \tightlist
  \item
    For example, the WCET for two possible paths might be calculated as:
    \(\text{wcet} = \max(19 + 14 + 10, 21 + 13 + 10) = 44\) clock
    cycles.
  \end{itemize}
\end{itemize}

\section{Integer Linear Programming
(ILP)}\label{integer-linear-programming-ilp}

\begin{itemize}
\item
  \textbf{ILP} is used for WCET analysis by defining an
  \textbf{objective function} to maximize execution time under given
  constraints (Slide 28-30).

  \begin{itemize}
  \tightlist
  \item
    The constraints are typically linear inequalities describing valid
    program transitions and flow consistency, allowing ILP to provide an
    upper bound for WCET.
  \end{itemize}
\end{itemize}

\section{Coding Guidelines for Time
Predictability}\label{coding-guidelines-for-time-predictability}

\begin{itemize}
\item
  \textbf{Bounded Loops}: Always use bounded loops to guarantee an upper
  bound on execution time (Slide 36). For example, in the \textbf{GCD}
  function, the loop is bounded to ensure a predictable runtime.
\item
  \textbf{Static Memory Allocation}: Prefer static allocation over
  \textbf{dynamic memory allocation} (e.g., \texttt{malloc}) to prevent
  runtime unpredictability caused by memory fragmentation or variable
  allocation times (Slide 38).
\item
  \textbf{Avoiding Interrupts}: Where possible, avoid
  \textbf{interrupts} due to their impact on execution time variability.
  Instead, use polling or reactive processors that can queue interrupts
  predictably (Slide 39).
\item
  \textbf{Fixed Point Arithmetic}: Use \textbf{fixed-point arithmetic}
  over floating-point for better predictability, as floating-point
  operations are not always supported natively and can have variable
  runtime characteristics.
\end{itemize}

\section{PRET Philosophy}\label{pret-philosophy}

\begin{itemize}
\tightlist
\item
  The \textbf{PRET (Precision Timed)} philosophy emphasizes treating
  time as a first-class citizen, aiming for a \textbf{time-predictable
  system} (Slide 35). The objective is to achieve predictability in
  hardware, software-hardware interfaces, and the software itself,
  ensuring reliable operation within a time-constrained environment.
\end{itemize}

\section{Conclusion}\label{conclusion-5}

This lecture provided insights into the design and analysis of
\textbf{embedded control systems}. Key topics included \textbf{real-time
system requirements}, \textbf{sampling and quantization},
\textbf{computation delays}, \textbf{WCET analysis}, and \textbf{coding
practices for time predictability}. Accurate timing analysis and
predictable software execution are crucial for the development of safe
and reliable embedded control systems. The concepts discussed lay the
foundation for designing robust control systems that meet strict
real-time requirements.

\chapter{Embedded Control Systems 2}\label{embedded-control-systems-2}

\section{Introduction to Embedded Control
Systems}\label{introduction-to-embedded-control-systems-1}

In this lecture, we focused on advanced topics in \textbf{embedded
control systems}, including \textbf{PWM control}, \textbf{stepper
motors}, \textbf{PID control}, and \textbf{ADC technologies}. These
components play a critical role in achieving accurate and efficient
control in real-time embedded applications (Slide 2).

\section{Pulse-Width Modulation (PWM)}\label{pulse-width-modulation-pwm}

\begin{itemize}
\item
  \textbf{Pulse-Width Modulation (PWM)} is a technique used to simulate
  an analog signal using digital means by varying the duty cycle of a
  digital pulse (Slide 5).

  \begin{itemize}
  \item
    \textbf{Duty Cycle}: The percentage of time the signal remains high
    during a cycle determines the average voltage output. A higher duty
    cycle implies higher average power delivered, which is crucial in
    controlling actuators like DC motors and LEDs (Slide 7).
  \item
    \textbf{Fast PWM vs Phase-Correct PWM}: There are different types of
    PWM modes. \textbf{Fast PWM} operates at a higher frequency to
    minimize flickering, while \textbf{Phase-Correct PWM} maintains
    symmetry in the signal waveform, which is often better for
    controlling motors with less noise (Slide 8).
  \end{itemize}
\end{itemize}

\section{DC and Stepper Motors}\label{dc-and-stepper-motors}

\subsection{DC Motors}\label{dc-motors}

\begin{itemize}
\item
  \textbf{DC Motors} are widely used for their simplicity and ease of
  control. Speed control is achieved by modulating the voltage applied
  to the motor, which is often implemented through PWM. The relationship
  between PWM duty cycle and motor speed is usually non-linear,
  requiring careful tuning to achieve precise control (Slide 10).
\item
  \textbf{Motor Driver Circuitry}: To drive DC motors,
  \textbf{H-bridges} are commonly used. An H-bridge allows for
  directional control by reversing the current flow through the motor,
  enabling forward and reverse motion (Slide 11).
\end{itemize}

\subsection{Stepper Motors}\label{stepper-motors}

\begin{itemize}
\item
  \textbf{Stepper Motors} are used when precise control of angular
  position is required. Each pulse sent to the motor corresponds to a
  fixed angular movement, allowing fine control over its rotation (Slide
  12).

  \begin{itemize}
  \item
    \textbf{Types of Stepper Motors}: Two main types include
    \textbf{unipolar} and \textbf{bipolar} stepper motors, each with
    distinct wiring configurations and driving requirements (Slide 13).
  \item
    \textbf{Stepper Motor Control}: To control the stepper motor,
    sequences of pulses must be sent in the correct order to energize
    the windings and produce the desired stepping motion. This is
    typically achieved using a \textbf{stepper motor driver}, which
    abstracts the complexity of pulse timing (Slide 14).
  \end{itemize}
\end{itemize}

\section{PID Control}\label{pid-control}

\textbf{Proportional-Integral-Derivative (PID) Control} is a control
loop mechanism widely used in industrial control systems to achieve
desired system behavior.

\begin{itemize}
\item
  \textbf{Proportional Control (P)}: The \textbf{proportional} term
  produces an output value proportional to the current error value. If
  the error is large, the control response is also large. This helps in
  reducing the response time but may lead to instability if used alone
  (Slide 17).
\item
  \textbf{Integral Control (I)}: The \textbf{integral} term sums the
  error over time, addressing the \textbf{steady-state error} by
  accumulating corrections until the desired setpoint is achieved. This
  component helps eliminate any residual error that remains after the
  proportional action has been applied (Slide 18).
\item
  \textbf{Derivative Control (D)}: The \textbf{derivative} term
  considers the rate of change of the error, providing a damping effect
  that reduces overshoot. It predicts future error and helps the system
  respond more smoothly, avoiding large fluctuations (Slide 19).
\item
  \textbf{Tuning PID Parameters}: Tuning the PID parameters (\(K_p\),
  \(K_i\), and \(K_d\)) is crucial for optimal system performance.
  Methods like \textbf{Ziegler-Nichols} can be used to empirically
  determine the ideal values, balancing response speed, stability, and
  accuracy (Slide 20).
\end{itemize}

\section{Analog-to-Digital Conversion
(ADC)}\label{analog-to-digital-conversion-adc}

\begin{itemize}
\item
  \textbf{Analog-to-Digital Converters (ADC)} are used to digitize
  analog signals so that they can be processed by digital controllers.
  ADCs are critical in embedded control systems where signals like
  temperature, pressure, or speed need to be sampled and processed
  (Slide 22).

  \begin{itemize}
  \item
    \textbf{Quantization and Resolution}: The resolution of an ADC
    refers to the number of discrete levels used to represent the analog
    input. A higher resolution means better accuracy but increased cost
    and complexity. For instance, a 12-bit ADC provides \(2^{12}\)
    quantization levels, which significantly improves signal
    representation over an 8-bit ADC (Slide 23).
  \item
    \textbf{Sampling Rate}: The sampling rate must be at least twice the
    highest frequency of the input signal to prevent \textbf{aliasing}
    (Nyquist criterion). If the sampling rate is insufficient, aliasing
    can lead to distorted signals that compromise control accuracy
    (Slide 24).
  \end{itemize}
\item
  \textbf{Types of ADC Technologies}:

  \begin{itemize}
  \item
    \textbf{Successive Approximation Register (SAR)}: A commonly used
    ADC type that offers a good balance between speed and accuracy. The
    SAR ADC works by iteratively approximating the input signal, which
    makes it suitable for medium-speed applications (Slide 25).
  \item
    \textbf{Sigma-Delta ADC}: Provides very high resolution but at the
    cost of slower conversion speed, often used in applications where
    precision is more critical than speed, such as audio signal
    processing (Slide 26).
  \item
    \textbf{Flash ADC}: The fastest type of ADC, using a parallel set of
    comparators to convert the analog signal in a single step. However,
    it requires a large number of comparators, making it expensive and
    power-hungry, suitable for high-speed applications (Slide 27).
  \end{itemize}
\end{itemize}

\section{Practical Considerations in Embedded Control
Systems}\label{practical-considerations-in-embedded-control-systems}

\begin{itemize}
\item
  \textbf{Noise Handling}: Analog signals are often susceptible to
  noise. Techniques such as \textbf{filtering} and \textbf{shielded
  cabling} are used to mitigate noise effects, ensuring that the ADC
  accurately captures the intended signal without distortion (Slide 29).
\item
  \textbf{Anti-Aliasing Filters}: To prevent aliasing, \textbf{low-pass
  filters} are often used before the ADC to remove high-frequency
  components that could interfere with accurate sampling (Slide 30).
\end{itemize}

\section{Case Study: Motor Speed
Control}\label{case-study-motor-speed-control}

\begin{itemize}
\item
  \textbf{Objective}: The case study presented a practical example of
  motor speed control using a combination of PWM and PID control (Slide
  32).
\item
  \textbf{Implementation}: The motor speed is monitored using an
  \textbf{optical encoder}, which provides feedback to the PID
  controller. The PID controller adjusts the PWM duty cycle to minimize
  the speed error and maintain the desired speed despite varying load
  conditions (Slide 33).
\item
  \textbf{Challenges}: Tuning the PID parameters was a challenge due to
  the non-linearities in motor response and the presence of external
  disturbances. Real-world control systems often require iterative
  tuning to balance response time, overshoot, and stability (Slide 34).
\end{itemize}

\section{Conclusion}\label{conclusion-6}

This lecture expanded on the foundational knowledge of embedded control
systems by introducing practical components like \textbf{PWM},
\textbf{stepper motors}, \textbf{PID controllers}, and \textbf{ADCs}.
Understanding these components and their interactions is essential for
designing robust and efficient control systems. Future lectures will
focus on more advanced topics such as \textbf{real-time operating
systems (RTOS)}, \textbf{sensor fusion}, and \textbf{model predictive
control (MPC)}, which are critical for modern, complex embedded control
applications.

\chapter{Embedded Control Systems 3}\label{embedded-control-systems-3}

\section{Introduction to Control
Systems}\label{introduction-to-control-systems}

In this lecture, we delve deeper into \textbf{control systems}, focusing
on both \textbf{open-loop} and \textbf{closed-loop} configurations, as
well as the terminology, benefits, and challenges associated with each
approach (Slide 3). Control systems are used to ensure that the output
of a physical system tracks a desired reference value by manipulating
the inputs to the system. This lecture also highlighted the importance
of stability, performance, and disturbance rejection in control systems.

\begin{tcolorbox}[enhanced jigsaw, title=\textcolor{quarto-callout-important-color}{\faExclamation}\hspace{0.5em}{Important Note}, breakable, opacityback=0, opacitybacktitle=0.6, leftrule=.75mm, left=2mm, coltitle=black, colback=white, toptitle=1mm, titlerule=0mm, arc=.35mm, rightrule=.15mm, colbacktitle=quarto-callout-important-color!10!white, toprule=.15mm, bottomtitle=1mm, colframe=quarto-callout-important-color-frame, bottomrule=.15mm]

\section*{\texorpdfstring{\textbf{THIS IS NOT THE SAME AS ELECTENG 332.
THIS IS A DIFFERENT
COURSE.}}{THIS IS NOT THE SAME AS ELECTENG 332. THIS IS A DIFFERENT COURSE.}}\label{this-is-not-the-same-as-electeng-332.-this-is-a-different-course.}

\markright{\textbf{THIS IS NOT THE SAME AS ELECTENG 332. THIS IS A
DIFFERENT COURSE.}}

\end{tcolorbox}

\section{Open-Loop vs Closed-Loop Control
Systems}\label{open-loop-vs-closed-loop-control-systems}

\subsection{Open-Loop Control Systems}\label{open-loop-control-systems}

\begin{itemize}
\item
  \textbf{Definition}: In an \textbf{open-loop control system}, the
  controller adjusts the inputs to the plant without considering the
  actual output (Slide 14). It does not measure how well the output
  matches the desired reference.
\item
  \textbf{Components}:

  \begin{itemize}
  \item
    \textbf{Plant}: The physical system to be controlled (e.g., a car, a
    disk drive, or a heater) (Slide 12).
  \item
    \textbf{Actuator}: A device that changes the parameters of the
    system, such as a throttle or motor.
  \item
    \textbf{Reference}: The desired value of the output, such as speed
    or temperature.
  \end{itemize}
\item
  \textbf{Characteristics}:

  \begin{itemize}
  \item
    \textbf{No Feedback}: Open-loop systems do not use feedback to
    correct their actions.
  \item
    \textbf{Feed-Forward Control}: There may be delays in actual output
    changes.
  \item
    \textbf{Predictable Output}: Open-loop systems are suitable when the
    plant output responds predictably to inputs and disturbances are
    minimal (Slide 14).
  \end{itemize}
\end{itemize}

\subsection{Closed-Loop Control
Systems}\label{closed-loop-control-systems}

\begin{itemize}
\item
  \textbf{Definition}: In a \textbf{closed-loop control system}, the
  controller continuously adjusts the inputs based on the difference
  between the measured output and the desired reference (Slides 15, 27).
\item
  \textbf{Components}:

  \begin{itemize}
  \item
    \textbf{Sensor}: Measures the plant output and provides feedback.
  \item
    \textbf{Error Detector}: Calculates the error by comparing the
    output with the reference value.
  \item
    \textbf{Feedback Loop}: This helps to adjust inputs to minimize the
    error.
  \end{itemize}
\item
  \textbf{Benefits}:

  \begin{itemize}
  \item
    \textbf{Error Minimization}: By using feedback, the closed-loop
    system reduces tracking errors.
  \item
    \textbf{Adaptation to Disturbances}: The system can react to
    changes, such as disturbances, to maintain desired performance.
  \item
    \textbf{Stability}: Ensures that all control variables remain
    bounded, ideally making the error converge to zero over time (Slide
    9).
  \end{itemize}
\end{itemize}

\subsection{Performance Metrics of Control
Systems}\label{performance-metrics-of-control-systems}

\begin{itemize}
\item
  \textbf{Rise Time}: The time taken for the output to go from 10\% to
  90\% of the desired value (Slide 10).
\item
  \textbf{Peak Time}: The time taken to reach the first peak of the
  response.
\item
  \textbf{Overshoot}: The percentage by which the output exceeds the
  final value.
\item
  \textbf{Settling Time}: The time it takes for the output to reach and
  stay within 1\% of the final value.
\item
  These metrics help assess how well the output tracks changes in the
  reference input and how the system responds to disturbances.
\end{itemize}

\section{Designing Control Systems}\label{designing-control-systems}

\subsection{Open-Loop Control Design}\label{open-loop-control-design}

\begin{itemize}
\item
  \textbf{Plant Modeling}: The plant model represents how the output
  changes based on inputs and the current state. Having a model helps in
  designing a better controller, although experimentation can be useful
  (Slide 18).

  \begin{itemize}
  \tightlist
  \item
    Example: In a car model, the speed is a function of the current
    speed and throttle position. The model can help predict how changes
    in the throttle angle affect the speed.
  \end{itemize}
\item
  \textbf{Controller Design}: A simple proportional controller can be
  used to manipulate the plant input. However, achieving the desired
  output requires tuning and careful consideration of system dynamics.
\end{itemize}

\subsection{Closed-Loop Control
Design}\label{closed-loop-control-design}

\begin{itemize}
\item
  \textbf{Proportional (P) Controller}: A proportional controller
  applies control by multiplying the tracking error by a gain value
  (Slide 28).

  \begin{itemize}
  \tightlist
  \item
    \textbf{Effects}: Adjusting the gain value (Κ) affects transient
    response, steady-state tracking, and disturbance rejection. However,
    increasing the gain can lead to oscillations (Slides 29-31).
  \end{itemize}
\item
  \textbf{Derivative (D) Controller}: Adds a control action based on the
  rate of change of the error, helping reduce overshoot and speed up
  convergence to the desired output (Slides 44-45).
\item
  \textbf{Integral (I) Controller}: Addresses steady-state error by
  accumulating the error over time, ensuring that the output reaches the
  desired value eventually. However, it may introduce oscillations if
  not carefully tuned (Slides 49-50).
\item
  \textbf{PID Controller}: A combination of Proportional, Integral, and
  Derivative components, the PID controller offers good steady-state
  performance, minimizes oscillations, and improves disturbance
  rejection (Slide 52).
\end{itemize}

\section{Example Analysis of
Controllers}\label{example-analysis-of-controllers}

\begin{itemize}
\item
  \textbf{Open-Loop vs Closed-Loop Performance}: Open-loop controllers
  struggle to handle disturbances, whereas closed-loop controllers can
  mitigate the effects of changes in the environment (Slide 42).
\item
  \textbf{PD Controller}: By incorporating derivative action, the PD
  controller improves transient response, but does not change
  steady-state error (Slide 46).
\item
  \textbf{PI Controller}: Useful for eliminating steady-state error by
  continually adjusting based on accumulated error (Slide 49).
\item
  \textbf{PID Controller}: The combined PID control offers comprehensive
  benefits by balancing response time, minimizing steady-state error,
  and controlling oscillations.
\end{itemize}

\section{Stability and Oscillation}\label{stability-and-oscillation}

\begin{itemize}
\item
  \textbf{Stability}: The stability of the control system is a critical
  objective, ensuring all control variables remain bounded (Slide 32).

  \begin{itemize}
  \item
    The rate of decay of the effect of initial speed is controlled by
    the gain value (\(\alpha\)). Setting a suitable gain is essential to
    avoid unbounded growth or excessive oscillations (Slides 33-34).
  \item
    \textbf{Oscillation}: Negative values of the rate of decay parameter
    (\(\alpha\)) can lead to oscillatory behavior, which must be avoided
    in most practical control applications (Slide 38).
  \end{itemize}
\end{itemize}

\section{Trade-Offs in Controller
Design}\label{trade-offs-in-controller-design}

\begin{itemize}
\tightlist
\item
  \textbf{Proportional Gain (Κ)}: Higher values can improve response
  time but lead to increased oscillations.
\item
  \textbf{Disturbance Rejection}: To effectively reject disturbances,
  the proportional gain must be balanced to avoid compromising system
  stability (Slide 34).
\item
  \textbf{Performance Objectives}: Reducing oscillation, achieving
  faster convergence, and minimizing steady-state error are often
  conflicting objectives, requiring careful tuning and consideration of
  the application requirements (Slide 40).
\end{itemize}

\section{Conclusion}\label{conclusion-7}

This lecture provided an overview of \textbf{open-loop} and
\textbf{closed-loop control systems}, the key components involved, and
the trade-offs between stability, response time, and error minimization.
Understanding these concepts is crucial for designing robust and
efficient control systems, whether using simple proportional control or
more advanced \textbf{PID controllers}. Future discussions will explore
more advanced control techniques, such as \textbf{state-space models}
and \textbf{model predictive control (MPC)}, which are particularly
useful for managing complex, multi-variable systems.

\chapter{Industrial Automation 1}\label{industrial-automation-1}

\section{Introduction to Industrial
Automation}\label{introduction-to-industrial-automation}

\textbf{Industrial Automation} (IA) refers to the use of computers,
programmable logic controllers (PLCs), and other control systems to
operate industrial machinery and processes without requiring human
intervention (Slide 5). This step beyond mechanization aims to improve
productivity, quality, flexibility, and safety while reducing
operational costs and human errors.

\section{Advantages and Drawbacks of Industrial
Automation}\label{advantages-and-drawbacks-of-industrial-automation}

\subsection{Advantages}\label{advantages}

\begin{itemize}
\tightlist
\item
  \textbf{Productivity}: Automated systems can operate continuously,
  24/7, significantly improving output compared to human-operated
  systems.
\item
  \textbf{Quality and Waste Reduction}: Automation reduces the
  likelihood of human error, leading to better product quality and less
  waste. Sensors provide accurate, quantitative data, which contributes
  to enhanced consistency.
\item
  \textbf{Flexibility}: Machines can be easily reprogrammed to handle
  new tasks, making automation well-suited for dynamic production
  environments where frequent changes are required.
\item
  \textbf{Information Accuracy}: Automation employs sensors to gather
  quantitative data, reducing errors that arise from subjective human
  measurements.
\item
  \textbf{Safety}: By automating dangerous tasks, robots and other
  automated equipment can work in environments that are unsafe for human
  workers, such as those involving hazardous chemicals or extreme
  temperatures.
\end{itemize}

\subsection{Drawbacks}\label{drawbacks}

\begin{itemize}
\tightlist
\item
  \textbf{Initial Cost}: Setting up industrial automation systems
  involves high initial investment costs, including purchasing robots
  and making changes to existing production lines.
\item
  \textbf{Maintenance}: Ongoing maintenance is required to ensure that
  automated systems operate effectively. The need for specialized
  maintenance plans can increase overall costs.
\end{itemize}

\section{Basic Components of Industrial
Automation}\label{basic-components-of-industrial-automation}

\begin{itemize}
\tightlist
\item
  \textbf{Programmable Logic Controllers (PLCs)}: Hardened computers
  specifically designed for industrial use to control processes,
  machinery, and systems (Slide 8).
\item
  \textbf{Numerical Controllers and Motion Controllers}: These devices
  are used to control specific tasks, such as the movement of mechanical
  parts.
\item
  \textbf{Remote I/O and Communication Devices}: These components are
  essential for enabling communication between different parts of the
  automation system.
\item
  \textbf{Industrial PCs}: Used for tasks that require more
  computational power or data processing capabilities than PLCs can
  provide.
\item
  \textbf{Mechatronic Components}: Include drives, sensors, actuators,
  valves, etc., which interact with physical systems to achieve
  automation objectives.
\end{itemize}

\section{PLCs and HMIs}\label{plcs-and-hmis}

\begin{itemize}
\tightlist
\item
  \textbf{PLCs}: PLCs are at the core of industrial automation,
  providing fast and reliable control. They are capable of synchronizing
  input from sensors and output to actuators, allowing for precise
  control of machinery (Slide 9).

  \begin{itemize}
  \tightlist
  \item
    \textbf{Characteristics of PLCs}: They are \textbf{fast} (providing
    response times in milliseconds), \textbf{durable} (built to
    withstand tough industrial conditions), \textbf{maintainable} (can
    be programmed using ladder logic), and \textbf{reliable} due to
    their simplicity and robustness (Slide 11).
  \end{itemize}
\item
  \textbf{Human-Machine Interface (HMI)}: HMIs are used to interact with
  PLCs, allowing operators to input and monitor system parameters, such
  as temperature or pressure settings (Slide 9).
\end{itemize}

\section{Generations of Industrial
Automation}\label{generations-of-industrial-automation}

\subsection{Generation 1: Relay Ladder Circuits (Slide
15)}\label{generation-1-relay-ladder-circuits-slide-15}

\begin{itemize}
\tightlist
\item
  \textbf{Hard-Wired Logic}: Early automation used hard-wired ladder
  logic circuits to control industrial equipment. These circuits were
  inflexible and required extensive rewiring to make changes.
\end{itemize}

\subsection{Generation 2: Programmable Logic Controllers (PLCs) (Slide
16)}\label{generation-2-programmable-logic-controllers-plcs-slide-16}

\begin{itemize}
\tightlist
\item
  \textbf{Introduction of PLCs}: PLCs revolutionized automation by
  providing programmable, hardened computers that replaced hard-wired
  circuits. They significantly improved the flexibility of automation
  systems, allowing for easier modification and expansion.
\end{itemize}

\subsection{Generation 3: Multifunctional PLCs (Slide
17)}\label{generation-3-multifunctional-plcs-slide-17}

\begin{itemize}
\tightlist
\item
  \textbf{Modern PLCs}: Today's PLCs support multiple programming
  languages and feature a versatile architecture. They can be expanded
  and integrated into larger systems, providing more functionality
  beyond simple logic control.
\end{itemize}

\subsection{Generation 4: Distributed Automation Solutions (Slide
20)}\label{generation-4-distributed-automation-solutions-slide-20}

\begin{itemize}
\tightlist
\item
  \textbf{Distributed Systems}: Modern automation relies on distributed
  systems where a central processor interacts with sensors and actuators
  via networked field-area buses. This architecture offers enhanced
  scalability and modularity, enabling more complex automation
  solutions.
\end{itemize}

\section{Challenges in Industrial
Automation}\label{challenges-in-industrial-automation}

\subsection{Challenge 1: Flexible Modular Manufacturing (Slide
23)}\label{challenge-1-flexible-modular-manufacturing-slide-23}

\begin{itemize}
\tightlist
\item
  \textbf{Flexibility in Manufacturing}: The key challenge is to replace
  components of the manufacturing system without affecting the rest of
  the system. A modular approach enables seamless replacement and
  modification of conveyor sections or robotic workstations without the
  need for significant system reconfiguration.
\end{itemize}

\subsection{Challenge 2: Dynamic Self-Configurability (Slide
27)}\label{challenge-2-dynamic-self-configurability-slide-27}

\begin{itemize}
\tightlist
\item
  \textbf{Self-Configurable Systems}: The dream of automation engineers
  is a system that can dynamically adapt and self-configure to handle
  different tasks without manual intervention. Achieving this level of
  automation requires sophisticated communication, middleware, and
  intelligent control systems.
\end{itemize}

\section{Object-Oriented Design of Automation
Software}\label{object-oriented-design-of-automation-software}

\begin{itemize}
\tightlist
\item
  \textbf{Object-Oriented Design}: Designing software to match the
  physical structure of automation systems enhances modularity and
  flexibility. For example, structuring software components to represent
  individual physical units enables easy modification and adaptation
  (Slide 21).
\end{itemize}

\section{Intelligent Agent-Based
Automation}\label{intelligent-agent-based-automation}

\begin{itemize}
\tightlist
\item
  \textbf{Agents in Automation}: The future of automation lies in using
  intelligent agents that can negotiate, recognize situations, make
  decisions, and coordinate operations autonomously (Slides 28-29).

  \begin{itemize}
  \tightlist
  \item
    \textbf{Agent Architecture}: The architecture involves controllers,
    processes, and agents that can autonomously interact and make
    decisions. This helps achieve more resilient and adaptive industrial
    processes.
  \end{itemize}
\end{itemize}

\section{Conclusion}\label{conclusion-8}

\textbf{Industrial Automation} has undergone significant evolution, from
hard-wired relay circuits to modern distributed and agent-based
solutions. Each generation has brought greater flexibility, efficiency,
and reliability to automation systems (Slide 30). Despite the
challenges, such as the high initial cost and maintenance demands,
industrial automation continues to advance, driven by the need for
enhanced productivity, quality, and safety. The next stages will focus
on improving self-configurability, modularity, and integrating
intelligent, autonomous systems to enable seamless, adaptive
manufacturing environments.

\chapter{Industrial Automation 2}\label{industrial-automation-2}

\section{Introduction to Industrial Automation
Systems}\label{introduction-to-industrial-automation-systems}

This lecture focused on advanced elements of industrial automation,
covering concepts such as the \textbf{IEC 61131 standard},
\textbf{Ladder Logic}, and control using \textbf{counters} and
\textbf{timers}. These components are foundational in designing robust
and flexible automated systems for industry (Slide 2).

\section{IEC 61131 Standard}\label{iec-61131-standard}

\textbf{IEC 61131} is an international standard for programmable logic
controllers (PLCs) that defines programming languages and structures
used for industrial control (Slide 2). It supports a variety of
programming approaches, including: - \textbf{Ladder Logic (LD)}: A
graphical programming language that uses symbols to represent electrical
relay logic, making it easy for engineers to visualize and debug control
processes. - \textbf{Instruction List (IL)}: A low-level textual
language similar to assembly, useful for describing control operations
in a concise manner (Slide 3). - \textbf{Structured Text (ST)}: A
high-level, block-structured language similar to traditional programming
languages, which makes it ideal for complex logical operations (Slide
3). - \textbf{Function Block Diagrams (FBD)} and \textbf{Sequential
Function Charts (SFC)}: Graphical representations that show the flow of
control and parallel operations (Slide 3).

\section{Ladder Logic and Its
Elements}\label{ladder-logic-and-its-elements}

\textbf{Ladder Logic} represents control processes in a form that
resembles a ladder. It has been used since the 1940s and is still
popular today because of its visual representation, ease of debugging,
and intuitive structure (Slides 4-7).

\begin{itemize}
\tightlist
\item
  \textbf{Basic Elements}:

  \begin{itemize}
  \tightlist
  \item
    \textbf{Contacts}: Represented as normally open or normally closed
    switches, contacts control the flow of ``power'' through each rung
    of the ladder diagram (Slide 5).
  \item
    \textbf{Coils}: Act as outputs, controlling devices like motors or
    valves based on the conditions specified by the contacts (Slide 5).
  \item
    \textbf{Connectors}: Elements arranged either in \textbf{series}
    (AND logic) or \textbf{parallel} (OR logic) to determine the
    conditions under which outputs are activated (Slide 5).
  \end{itemize}
\item
  \textbf{Ladder Logic Conventions}:

  \begin{itemize}
  \tightlist
  \item
    \textbf{Power Rails}: Vertical lines representing power flow from
    left to right across each ``rung'' of the ladder (Slide 6).
  \item
    \textbf{Rungs}: Each rung represents an operation in the control
    process. Outputs are updated only after evaluating all inputs in the
    program cycle, ensuring deterministic behavior (Slide 6).
  \item
    Devices are shown in their \textbf{normal condition} (inactive or
    off state) and can appear in multiple rungs as part of different
    control sequences (Slide 6).
  \end{itemize}
\end{itemize}

\section{Example: Pneumatic Cylinder
Control}\label{example-pneumatic-cylinder-control}

A simple example of using ladder logic to control a \textbf{pneumatic
cylinder} was provided, showcasing its applicability in real-world
automation tasks: - \textbf{Desired Behavior}: An LED indicates
readiness. When a ``START'' button is pressed, the cylinder moves
forward until fully extended, then returns to its default position.
Sensors provide signals for ``HOME'' (default position) and ``END''
(fully extended) states (Slide 10). - \textbf{State Machine
Representation}: The control process can be described using a
\textbf{Moore state machine}, where inputs trigger state transitions and
determine outputs. Boolean logic defines the state transitions, ensuring
that the system follows a predictable sequence of actions (Slide 13).

\section{Ladder Logic Timers and
Counters}\label{ladder-logic-timers-and-counters}

\textbf{Timers} and \textbf{counters} are used in ladder logic for
managing time-dependent operations and counting events, such as sensor
activations or production cycles.

\subsection{Timers}\label{timers}

\begin{itemize}
\tightlist
\item
  \textbf{On-Delay Timer (TON)}: The output turns on after a specified
  delay time once the input signal is activated. This is used for
  operations that need a fixed delay before activation, such as starting
  a motor after a set period (Slide 28).
\item
  \textbf{Off-Delay Timer (TOF)}: The output turns off after a delay
  once the input signal is deactivated. This is useful for keeping an
  actuator active for a fixed period after the input stops (Slide 30).
\item
  \textbf{Pulse Timer (TP)}: Produces a fixed-length output pulse when
  the input is activated. This is often used in signaling applications
  where a precise pulse duration is required (Slide 33).
\end{itemize}

\subsection{Counters}\label{counters}

\begin{itemize}
\tightlist
\item
  \textbf{Up-Counters (CTU)}: Count up from zero each time a rising edge
  of an input is detected until a preset value is reached (Slide 23).
\item
  \textbf{Down-Counters (CTD)}: Count down from a preset value to zero,
  commonly used for decrementing remaining quantities, such as material
  in a hopper.
\item
  \textbf{Resetting Counters}: A reset input brings the counter value
  back to zero, ensuring it starts counting again from the beginning
  (Slide 24).
\end{itemize}

\section{Sequential Function Chart
(SFC)}\label{sequential-function-chart-sfc}

\textbf{Sequential Function Charts (SFC)} are another important tool for
industrial automation, providing a high-level graphical representation
of a system's operational sequence. SFCs show the flow from one step to
another, detailing the conditions for each transition and enabling a
clear view of parallel and sequential operations (Slide 39). -
\textbf{Parallel Branches}: Allow multiple operations to occur
simultaneously, synchronized by transition conditions. These branches
are visually represented using double horizontal lines, making it easy
to identify concurrent activities in the system (Slide 41).

\section{Practical Considerations for Ladder
Logic}\label{practical-considerations-for-ladder-logic}

\begin{itemize}
\tightlist
\item
  \textbf{Why Ladder Logic Is Still Popular}: Ladder logic is favored
  due to its \textbf{ease of use}, \textbf{visual clarity}, and the
  ability to \textbf{trace command flow} intuitively. It is particularly
  useful for maintenance personnel without deep programming knowledge,
  as its representation is similar to traditional relay-based control
  systems (Slide 22).
\item
  \textbf{Counters and Timers}: Practical examples include
  \textbf{traffic light controllers} using timers to manage the sequence
  of lights, and counters to keep track of vehicles or system
  operations, ensuring safe and efficient process flows (Slide 36).
\end{itemize}

\section{Conclusions}\label{conclusions}

This lecture provided an in-depth overview of the tools and techniques
used in \textbf{industrial automation}, such as \textbf{Ladder Logic},
\textbf{SFC}, \textbf{IEC 61131} programming languages, and the use of
\textbf{counters} and \textbf{timers}. Understanding these concepts is
crucial for implementing reliable, flexible, and maintainable automation
systems. These methods allow for both the visual representation and
precise control of industrial processes, making them fundamental for
modern manufacturing environments (Slide 44).

In future discussions, we'll explore more advanced features of
automation systems, such as integrating \textbf{networked embedded
systems} and \textbf{agent-based control}, to further enhance
flexibility and scalability in industrial applications.




\end{document}
