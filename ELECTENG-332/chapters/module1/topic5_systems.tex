% !TeX root = ..\..\notes-main.tex
\documentclass[../notes-main.tex]{subfiles}
\begin{document}
\section{Topic 5: What are Systems?}
The term \emph{system} is derived from the Greek word \emph{systema}, which means an organised relationship among functioning units or components. It is often used to describe any orderly arrangement of ideas or constructs.\\
According to the Webster's Dictionary,
\begin{figure}[H]
    \centering
    \begin{mdframed}
        \begin{center}
            \textcolor{blue}{%
                \emph{``A system is an aggregation or assemblage of objects united by some form of regular interaction or interdependence; a group of diverse units so combined by nature or art as to form an integral; whole and to function, operate, or move in unison and often in obedience to some form of control\dots''}}
        \end{center}
    \end{mdframed}\label{fig:system-def-1}
    \vspace{-1em}
\end{figure}
\noindent According to the International Council on Systems Engineering (INCOSE),
\begin{figure}[H]
    \centering
    \begin{mdframed}
        \begin{center}
            \textcolor{red}{%
                \emph{A system is an arrangement of parts or elements that together exhibit behaviour or meaning that the individual constituents do not.}}\\ The elements or parts, can include people, hardware, software, facilities, policies, and documents; that is, all things required to produce system-level results.
        \end{center}
    \end{mdframed}\label{fig:system-def-2}
    \vspace{-1em}
\end{figure}
\noindent It is difficult to give a single and precise definition of the term \emph{system}, which will suit to different perspectives of different people. In practice, what is meant by ``the system'' depends on the objectives of a particular study.\\
From the control engineering perspective, \textcolor{red}{the system is any interconnection of components to achieve desired objectives.} It is characterised by its \textbf{\textcolor{blue}{inputs}}, \textbf{\textcolor{blue}{outputs}}, and the rules of operations or laws. For example:
\begin{itemize}
    \item[\textcolor{blue}{a.}] The laws of operation in electrical systems are Ohm's law, which gives the voltage-current relationships for resistors, capacitors and inductors, and Kirchhoff's laws, which govern the laws of interconnection of various electrical components.
    \item[\textcolor{blue}{b.}] Similarly, in mechanical systems, the laws of operation are Newton's laws. These laws can be used to derive mathematical models of the system.
\end{itemize}

\subsection{System as an Operator}
\begin{figure}[H]
    \centering
    \begin{mdframed}
        \begin{center}
            A system is defined mathematically \textcolor{red}{as a transformation which maps an input signal \(x(t)\) to an output signal \(y(t)\)} as shown in Figure:~\ref{fig:system-operator-def-2}. For a continuous time system, the input-output mapping is expressed as:
            \begin{equation}
                \textcolor{red}{y(t) = \mathcal{S}[x(t)]}, \quad \text{where} \; \mathcal{S} \; \text{is an operator.}
                \label{eq:system-operator-def-1}
            \end{equation}
        \end{center}
    \end{mdframed}\label{fig:system-operator-def-1}
    \vspace{-1em}
\end{figure}
\vspace{-1em}
\begin{figure}[H]
    \centering
    \tikzsetnextfilename{system_operator_diagram}
    \begin{mdframed}
        \begin{center}
            \begin{tikzpicture}[auto, thick, node distance=3cm, >=Latex]
                \node (input) at (0,0) {};
                \node [draw=borderblue, rectangle, minimum height=2em, minimum width=3em, right=of input] (system) {\large \(\mathcal{S}\)};
                \node [right=of system] (output) {};
                \draw [->] (input) -- node[above,midway] {\small\(x(t)\)} (system);
                \draw [->] (system) -- node[above,midway] {\small\(y(t) = \mathcal{S}[x(t)]\)} (output);
            \end{tikzpicture}
        \end{center}
    \end{mdframed}
    \vspace{-1em}\caption{System as an Operator}\label{fig:system-operator-def-2}
\end{figure}
\vspace{-1em}
\begin{figure}[H]
    \centering
    \begin{mdframed}
        \begin{center}
            \textcolor{red}{%
                A control system may be defined as an interconnection of components which are configured to provide a desired response.}
        \end{center}
    \end{mdframed}\label{fig:system-interconnect-def-1}
    \vspace{-1em}
\end{figure}
\subsection{Classification of Systems}
The basis of classifying systems are many. They can be classified according to the following:
\begin{itemize}
    \item[\textcolor{blue}{a.}] \textbf{The Time Frame:} (\emph{discrete, continuous or hybrid});
    \item[\textcolor{blue}{b.}] \textbf{System Complexity:} (\emph{physical, conceptual and esoteric});
    \item[\textcolor{blue}{c.}] \textbf{Uncertainties:} (\emph{deterministic and stochastic});
    \item[\textcolor{blue}{d.}] \textbf{Nature and type of components:} (\emph{static or dynamic, linear or nonlinear, time-invariant or time variant, lumped or distributed etc});
\end{itemize}
\begin{figure}[H]
    \centering
    \begin{mdframed}
        \begin{itemize}
            \item Linear and nonlinear systems;
            \item Time-invariant and time-variant systems;
            \item Static (memory less) and dynamic (with memory) systems;
            \item Causal and Non-causal systems;
            \item Lumped and distributed parameter systems;
            \item Deterministic and stochastic systems;
            \item Continuous and discrete systems;
        \end{itemize}
    \end{mdframed}\label{fig:system-type-list-1}
    \vspace{-1em}
\end{figure}
\vspace{-1em}
\subsection{Linear and Nonlinear Systems}
\begin{figure}[H]
    \centering
    \begin{mdframed}
        \begin{center}
            A system is said to be linear provided it satisfies the \textcolor{red}{\emph{principle of superposition}} which is the combination of the \textcolor{blue}{\emph{additive}} and \textcolor{blue}{\emph{homogeneity}} properties. Otherwise, it is \emph{non-linear}
        \end{center}
    \end{mdframed}\label{fig:linear-nonlinear-system-def-1}
    \vspace{-1em}
\end{figure}
\subsubsection{\textcolor{red}{Principle of Additivity:}} Assume the system initially at rest.
\begin{enumerate}[label=\blacktriangleright, leftmargin=*, itemsep=0.5em]
    \item Suppose an input \(x_1(t)\) to this system produces an output \(y_1(t)\) and an input \(x_2(t)\) produces an output \(y_2(t)\).
    \item If the system is linear, then the application of the input \(x_1(t) + x_2(t)\) will produce an output \(y_1(t) + y_2(t)\). Thus if
\end{enumerate}
\begin{equation}
    \textcolor{blue}{x_1(t) \rightarrow y_1(t) \quad \text{and} \quad x_2(t) \rightarrow y_2(t)}, \quad \text{then} \quad \textcolor{red}{x_1(t) + x_2(t) \rightarrow y_1(t) + y_2(t)}
    \label{eq:linear-system-additive-property}
\end{equation}
\begin{figure}[H]
    \centering
    \tikzsetnextfilename{additive_property_diagram}
    \begin{mdframed}
        \begin{center}
            \begin{tikzpicture}[auto, thick, node distance=2cm, >=Latex]
                \node (input1) at (0,0) {};
                \node [draw=borderblue, rectangle, minimum height=2em, minimum width=3em, right=of input1] (system1) {\large \(\mathcal{S}\)};
                \node [right=of system1] (output1) {};
                \node (input2) [below=1cm of input1] {};
                \node [draw=borderblue, rectangle, minimum height=2em, minimum width=3em, right=of input2] (system2) {\large \(\mathcal{S}\)};
                \node [right=of system2] (output2) {};
                \node [below=0.45cm of output1] (midpoint_systems) {};
                \node [right=0.01cm of midpoint_systems] (additivity_start){};
                \node [right=4cm of additivity_start] (additivity_end){};
                \node (input_combined) [right=0.01cm of additivity_end] {};
                \node [draw=borderblue, rectangle, minimum height=2em, minimum width=3em, right=of input_combined] (system_combined) {\large \(\mathcal{S}\)};
                \node [right=of system_combined] (output_combined) {};
                \draw [->] (input1) -- node[above,midway] {\(x_1(t)\)} (system1);
                \draw [->] (system1) -- node[above,midway] {\(y_1(t)\)} (output1);
                \draw [->] (input2) -- node[above,midway] {\(x_2(t)\)} (system2);
                \draw [->] (system2) -- node[above,midway] {\(y_2(t)\)} (output2);
                \draw [->] (input_combined) -- node[above,midway] {\scriptsize\(x_1(t) + x_2(t)\)} (system_combined);
                \draw [->] (system_combined) -- node[above,midway] {\scriptsize\(y_1(t) + y_2(t)\)} (output_combined);
                \draw [->, double equal sign distance] (additivity_start) -- node[above,midway] {\textbf{Additivity}} (additivity_end);
            \end{tikzpicture}
        \end{center}
    \end{mdframed}
    \vspace{-1em}\caption{Principle of Additivity}\label{fig:additive-property-def-2}
\end{figure}
\subsubsection{\textcolor{red}{Principle of Homogeneity or Scaling:}}
\begin{enumerate}[label=\blacktriangleright, leftmargin=*, itemsep=0.5em]
    \item Let an input \textcolor{red}{(\emph{cause})} \(x(t)\) produce an output \textcolor{red}{(\emph{effect})} \(y(t)\). If the system is linear then,
    \item Scaling the input (\emph{cause}) \(x(t)\) by a factor ``\(a\)'' will scale the output (\emph{effect}) \(y(t)\) by the same factor.
    \item Thus, if the \textcolor{blue}{input \(x(t)\) results in \(y(t)\)}, then the \textcolor{red}{scaled input \(ax(t)\)} gives the output \textcolor{red}{\(ay(t)\)} where ``\(a\)'' can either be a real or imaginary number. Thus, for a linear system:
\end{enumerate}
\begin{equation}
    \textcolor{red}{x(t) \rightarrow y(t) \quad \Rightarrow ax(t) \rightarrow ay(t)}
    \label{eq:linear-system-homogeneity-property}
\end{equation}
\begin{figure}[H]
    \centering
    \tikzsetnextfilename{homogeneity_property_diagram}
    \begin{mdframed}
        \begin{center}
            \begin{tikzpicture}[auto, thick, node distance=2cm, >=Latex]
                \node (input1) at (0,0) {};
                \node [draw=borderblue, rectangle, minimum height=2em, minimum width=3em, right of=input1] (system1) {\large \(\mathcal{S}\)};
                \node [right of=system1] (output1) {};
                \node (homogeneity_start) [right=0.1cm of output1] {};
                \node (homogeneity_end) [right=4cm of homogeneity_start] {};
                \node (input2) [right=0.01cm of homogeneity_end] {};
                \node [draw=borderblue, rectangle, minimum height=2em, minimum width=3em, right of=input2] (system2) {\large \(\mathcal{S}\)};
                \node [right of=system2] (output2) {};
                \draw [->] (input1) -- node[above,midway] {\(x(t)\)} (system1);
                \draw [->] (system1) -- node[above,midway] {\(y(t)\)} (output1);
                \draw [->] (input2) -- node[above,midway] {\(ax(t)\)} (system2);
                \draw [->] (system2) -- node[above,midway] {\(ay(t)\)} (output2);
                \draw [->, double equal sign distance] (homogeneity_start) -- node[above,midway] {\textbf{Homogeneity}} (homogeneity_end);
            \end{tikzpicture}
        \end{center}
    \end{mdframed}
    \vspace{-1em}\caption{Principle of Homogeneity}\label{fig:homogeneity-property-def-1}
\end{figure}
\newpage
\subsubsection{\textcolor{red}{Principle of Superposition:}}
This is the combination of both the additive and homogeneity properties for a linear system. (For those definitions, See Equations:~\ref{eq:linear-system-additive-property} and ~\ref{eq:linear-system-homogeneity-property}).
\begin{figure}[H]
    \centering
    \begin{mdframed}
        \begin{center}
            If the zero state response of a linear system due to a finite \(N\text{-number}\) of inputs \(x_1(t), x_2(t), \dots, x_N(t)\) equals to \(y_1(t), y_2(t), \dots, y_N(t)\) respectively, then the response of the system to the linear combination of these inputs
            \begin{equation}
                a_1x_1(t) + a_2x_2(t) + a_3x_3(t) + \dots\dots + a_Nx_N(t)
                \label{eq:linear-system-superposition-input}
            \end{equation}
            \noindent is given by the linear combination of the individual outputs i.e.
            \begin{equation}
                a_1y_1(t) + a_2y_2(t) + a_3y_3(t) + \dots\dots + a_Ny_N(t)
                \label{eq:linear-system-superposition-output}
            \end{equation}
            \noindent where \(a_1, a_2, a_3, \dots, a_N\) are arbitrary constants (either real or imaginary).   
            Thus, if 
            \begin{equation}
                \textcolor{blue}{x_1(t) \rightarrow y_1(t), \quad x_2(t) \rightarrow y_2(t), \dots, x_N(t) \rightarrow y_N(t)}
                \label{eq:linear-system-superposition-def-eq-1}
            \end{equation}
            then, for all \(a_1, a_2, \dots, a_N\)
            \begin{equation}
                \textcolor{blue}{a_1x_1(t) + a_2x_2(t) + \ldots  a_Nx_N(t)\rightarrow}\ \textcolor{red}{ a_1y_1(t) + a_2y_2(t) + \ldots  a_Ny_N(t)}
                \label{eq:linear-system-superposition-def-eq-2}
            \end{equation}          
        \end{center}
    \end{mdframed}\label{fig:linear-system-superposition-def-1}
    \vspace{-1em}
\end{figure}
\begin{figure}[H]
    \centering
    \tikzsetnextfilename{superposition_principle_diagram}
    \begin{mdframed}
        \begin{center}
            \begin{tikzpicture}[auto, thick, node distance=2cm, >=Latex]
                \node (input1) at (0,0) {};
                \node [draw=borderblue, rectangle, minimum height=2em, minimum width=3em, right=of input1] (system1) {\large \(\mathcal{S}\)};
                \node [right=of system1] (output1) {};
                \node [below=1cm of input1] (input2) {};
                \node [draw=borderblue, rectangle, minimum height=2em, minimum width=3em, right=of input2] (system2) {\large \(\mathcal{S}\)};
                \node [right=of system2] (output2) {};
                \node [below=1cm of input2] (inputN) {};
                \node [draw=borderblue, rectangle, minimum height=2em, minimum width=3em, right=of inputN] (systemN) {\large \(\mathcal{S}\)};
                \node [right=of systemN] (outputN) {};
                \draw [loosely dotted, very thick] (system2.south) -- (systemN.north);
                \node (superposition_start) [right=0.01cm of output2] {};
                \node (superposition_end) [right=4cm of superposition_start] {};
                \node (input_combined) [right=0.01cm of superposition_end] {};
                \node [draw=borderblue, rectangle, minimum height=2em, minimum width=3em, right=of input_combined] (system_combined) {\large \(\mathcal{S}\)};
                \node [right=of system_combined] (output_combined) {};
                \draw [->] (input1) -- node[above,midway] {\(x_1(t)\)} (system1);
                \draw [->] (system1) -- node[above,midway] {\(y_1(t)\)} (output1);
                \draw [->] (input2) -- node[above,midway] {\(x_2(t)\)} (system2);
                \draw [->] (system2) -- node[above,midway] {\(y_2(t)\)} (output2);
                \draw [->] (inputN) -- node[above,midway] {\(x_N(t)\)} (systemN);
                \draw [->] (systemN) -- node[above,midway] {\(y_N(t)\)} (outputN);
                \draw [->, double equal sign distance] (superposition_start) -- node[above,midway] {\textbf{Superposition}} (superposition_end);
                \draw [->] (input_combined) -- node[above,midway] {\footnotesize\(\sum_{i=1}^{N} a_i x_i(t)\)} (system_combined);
                \draw [->] (system_combined) -- node[above,midway] {\footnotesize\(\sum_{i=1}^{N} a_i y_i(t)\)} (output_combined);
            \end{tikzpicture}
                       
        \end{center}
    \end{mdframed}
    \vspace{-1em}\caption{Superposition Principle Diagram}\label{fig:superposition-principle-def-2}
\end{figure}
\vspace{-1em}
\newpage
\noindent In more mathematical terms:

\begin{figure}[H]
    \centering
    \begin{mdframed}
        
            Let \(\mathcal{S}\) be a linear system, and let \(x_1(t), x_2(t), \dots, x_N(t)\) be inputs with corresponding outputs \(y_1(t), y_2(t), \dots, y_N(t)\), such that \(\mathcal{S}(x_i(t)) = y_i(t)\) for each \(i \in \{1, 2, \dots, N\}\). Then, for any constants \(a_1, a_2, \dots, a_N \in \mathbb{C}\), the system's response to the linear combination of the inputs is given by the corresponding linear combination of the outputs.

            Formally:
            \begin{gather}
                \forall N \in \mathbb{N}, \\
                \forall \{a_i \in \mathbb{C}, x_i(t) \in \mathbb{C}, y_i(t) \in \mathbb{C}\}_{i=1}^N, \\
                \text{if} \ \mathcal{S}(x_i(t)) = y_i(t) \ \text{for all} \ i, \\
                \text{then} \ \mathcal{S}\left(\sum_{i=1}^{N} a_i x_i(t)\right) = \sum_{i=1}^{N} a_i y_i(t).
                \label{eq:linear-system-superposition-nerd}
            \end{gather}

            In this definition:
            \begin{itemize}
                \item \(\mathbb{N}\) denotes the set of natural numbers \(\{1,2,\dots\}\), representing the number of inputs;
                \item \(\mathbb{C}\) denotes the set of complex numbers, allowing for both real and imaginary components in the coefficients, inputs, and outputs;
                \item The system \(\mathcal{S}\) maps each input \(x_i(t)\) to its corresponding output \(y_i(t)\);
                \item The statement asserts that the linear combination of inputs yields the linear combination of outputs, which is the core of the superposition principle.
            \end{itemize}
        
    \end{mdframed}
    \vspace{-1em}\label{fig:linear-system-superposition-nerd-1}
\end{figure}
\subsection{Example: Linear and Nonlinear Systems}
Test the linearity/non-linearity of the system whose relation between the output \(y(t)\) and input \(x(t)\) is expressed as:
\begin{equation}
    y(t) = mx(t) + c
    \label{eq:linear-nonlinear-system-example-io-relation}
\end{equation}
\noindent where \(m\) is a constant.\\
\textbf{Solution:}
Let \(x_1(t)\) and \(x_2(t)\) be the two distinct inputs applied to the system. The outputs corresponding to these inputs are
\begin{equation}
    y_1(t) = mx_1(t) + c, \quad \& \quad y_2(t) = mx_2(t) + c
    \label{eq:linear-nonlinear-system-example-outputs}
\end{equation}
\noindent Let us apply an input, which is the linear combination of inputs \(x_1(t)\) and \(x_2(t)\) i.e.
\begin{equation}
    x(t) = ax_1(t) + bx_2(t)
    \label{eq:linear-nonlinear-system-example-input}
\end{equation}
\noindent The output due to this input is
\begin{equation}
    y(t) = m[ax_1(t) + bx_2(t)] + c \neq ay_1(t) + by_2(t)
\end{equation}
\noindent Hence, the system is \textcolor{red}{\emph{non-linear}}.

\subsection{Time-invariant and Time-varying Systems}
\begin{figure}[H]
    \centering
    \begin{mdframed}
        \begin{center}
            \textcolor{red}{\textbf{Definition:}} A system is \textcolor{blue}{time invariant} of a time shift in the input signal results in an identical time shift in the output signal. For example, in a time-invariant system, \textcolor{red}{if an input \(u(t)\) produces an output \(y(t)\), then application of a delayed input \(u(t - d)\) results in \(y(t - d)\)}. Mathematically, if
            \begin{gather}
                \textcolor{blue}{y(t) = \mathcal{S}[u(t)]}, \quad \text{then}\\ \textcolor{red}{y(t - d) = \mathcal{S}[u(t - d)]}
                \label{eq:time-invariant-system-def-1}
            \end{gather}
        \end{center}
    \end{mdframed}\label{fig:time-invariant-system-def-1}
    \vspace{-1em}
\end{figure}
\begin{figure}[H]
    \centering
    \begin{mdframed}
        \begin{center}
            \textcolor{red}{\textbf{Remark 1:}} If the input-output relation of a system is described by a linear differential equation of the form
            \begin{gather}
                \od[n]{y}{t} + a_1\od[n-1]{y}{t} + \dots + a_{n-1} \od{y}{t} + a_ny\\ 
                = b_0\od[m]{u}{t} + b_1\od[m-1]{u}{t} + \dots + b_{m-1}\od{u}{t} + b_mu
            \end{gather}
            then, this system is time invariant \textcolor{red}{if the coefficients \(a_i, i = 1,2,\ldots,n\) and \(b_i, i = 0,1,\ldots,m\) are constants} (i.e. if \(\forall a_i, \ i \in \mathbb{N}\ (\text{starting at 1})\ \& \ \forall b_i, \ i \in \mathbb{N}\ (\text{starting at 0})\)). If these coefficients are functions of time, then the system is a \textbf{linear time-varying} system.
        \end{center}
    \end{mdframed}\label{fig:time-varying-system-def-1}
    \vspace{-1em}
\end{figure}
\end{document}