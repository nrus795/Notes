% !TeX root = ..\..\notes-main.tex
% TODO Expand more on this topic, it's a bit too short on its own.
\documentclass[../notes-main.tex]{subfiles}
\begin{document}
\section{Topic 2: Concept of Engineering Infinity}
Consider a signal \(\expfn{-a}{t}\). The time constant for this signal is \(T = \frac{1}{a}\). Theoretically, the signal is meant to decay to zero as time approaches infinity, i.e.
\begin{equation}
    \lim_{t \to \infty} \expfn{-a}{t} = 0 
    \label{eq:eng-infinity}
\end{equation}
But in practice, this is not the case, as its value will be very, very small after five time constants \(5T\). This is the \redtext{\textbf{\emph{Concept of Engineering Infinity}}}. The signal will never reach zero, but it will be so small that it can be considered zero for all practical purposes. This is a very important concept in control systems, as it allows us to simplify our calculations and analysis.
\fxerror{Add more content to this topic.}
\end{document}