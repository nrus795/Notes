% !TeX root = ..\..\notes-main.tex
\documentclass[../notes-main.tex]{subfiles}
\begin{document}
\section{Topic 4: What are Signals?}
It is difficult to find a unique definition of a signal. However in the context of this course, we give a workable definition which suits most of our purposes as:
\begin{figure}[H]
    \centering
    \begin{mdframed}
        \begin{center}
            \textcolor{red}{%
                \emph{A signal conveys information about a physical phenomenon which evolves in time or space.}}
        \end{center}
    \end{mdframed}\label{fig:signal-def-1}
    \vspace{-1em}
\end{figure}
\vspace{-1em}
\noindent Examples of such signals include: Voltage, current, speech, television, images from remote space probes, voltages generated by the heart and brain, radar and sonar echoes, seismic vibrations, signals from GPS satellites, signals from human genes, and countless other applications.
\subsection{Energy \& Power Signals}
\textcolor{blue}{\textbf{Energy Signals}}
\vspace{-1em}
\begin{figure}[H]
    \centering
    \begin{mdframed}
        \begin{center}
            \textcolor{red}{%
                A signal is said to be an \emph{energy signal} if and only if it has finite energy.}
        \end{center}
    \end{mdframed}\label{fig:energy-signal-def-1}
    \vspace{-1em}
\end{figure}
\vspace{-1em}
\noindent \textcolor{red}{\textbf{Power Signals}}
\vspace{-1em}
\begin{figure}[H]
    \centering
    \begin{mdframed}
        \begin{center}
            \textcolor{blue}{%
                A signal is said to be a \emph{power signal} if and only if the average power of the signal is finite and non-zero.}
        \end{center}
    \end{mdframed}\label{fig:power-signal-def-1}
    \vspace{-1em}
\end{figure}
\vspace{-1em}
\noindent The instantaneous power \(p(t)\) of a signal \(x(t)\) is expressed as:
\begin{equation}
    \textcolor{red}{p(t) = x^2(t)}
    \label{eq:instantaneous-power-def}
\end{equation}
The total energy of a continuous-time signal \(x(t)\) is given by:
\begin{equation}
    \textcolor{blue}{E = \lim_{T \to \infty} \int_{-T/2}^{T/2} x^2(t) \dif t = \int_{-\infty}^{\infty} x^2(t) \dif t}
    \label{eq:continuous-time-energy-def}
\end{equation}
For a complex valued signal:
\begin{equation}
    \textcolor{red}{E = \int_{-\infty}^{\infty} \abs{x(t)}^2 \dif t}
    \label{eq:complex-signal-energy-def}
\end{equation}
Since power equals to the time average of the energy, the average power is given by:
\begin{equation}
    \textcolor{blue}{P = \lim_{T \to \infty} \frac{1}{T} \int_{-T/2}^{T/2} x^2(t) \dif t = \frac{E}{T}}
    \label{eq:average-power-def}
\end{equation}
\noindent Note that during calculation of energy, we average the power over an infinitely large interval.
\begin{figure}[H]
    \centering
    \begin{mdframed}
        \begin{center}
            \textcolor{red}{%
                A signal with finite energy has zero power and a signal with finite power has infinite energy.}
        \end{center}
    \end{mdframed}\label{fig:energy-power-relationship-1}
    \vspace{-1em}
\end{figure}
\vspace{-1em}
\begin{figure}[H]
    \centering
    \begin{mdframed}
        \begin{center}
            \begin{itemize}
                \item[\textcolor{blue}{a.}] A signal \textcolor{red}{can not both be an energy and a power signal.} This classification of signals based on power and energy are \textcolor{red}{mutually exclusive.}
                \item[\textcolor{blue}{b.}] However, \textcolor{blue}{a signal can belong to neither of the above two categories.}
                \item[\textcolor{blue}{c.}] The signals which are both deterministic and non-periodic have finite energy and therefore are energy signals.\,\textcolor{red}{Most of the signals, in practice, belong to this category.}
                \item[\textcolor{blue}{d.}] \textcolor{blue}{Periodic signals and random signals} are essentially \textcolor{red}{power signals.}
                \item[\textcolor{blue}{e.}] Periodic signals for which the area under \(\abs{x(t)}^2\) over one period is finite are power signals.
            \end{itemize}
        \end{center}
    \end{mdframed}\label{fig:energy-power-relationship-2}
    \vspace{-1em}
\end{figure}
\vspace{-1em}
\subsection{Examples}
\subsubsection{Example 1: Unit Step Function}
Consider a unit step function defined as:
\begin{equation}
    \textcolor{blue}{u(t) =
        \begin{cases}
            1 & t \geq 0         \\
            0 & \text{otherwise}
        \end{cases}}
        \label{eq:unit-step-function-example}
\end{equation}
Determine whether this is an energy signal or a power signal or neither.\\
\noindent \textbf{Solution:} Let us compute the energy of this signal as:
\begin{equation}
    E = \int_{-\infty}^{\infty} {[u(t)]}^2 \dif t = \int_{0}^{\infty} {[0]}^2 \dif t = \int_{0}^{\infty} {[1]}^2 \dif t = \infty
    \label{eq:unit-step-energy}
\end{equation}
Since the energy of this signal is infinite, it cannot be an energy signal. Let us compute the power of this signal as:
\begin{equation}
    \textcolor{blue}{P = \lim_{T \to \infty} \frac{1}{T} \int_{-T/2}^{T/2} {[u(t)]}^2 \dif t = \lim_{T \to \infty} \frac{1}{T} \int_{0}^{T/2} {[u(t)]}^2 \dif t = \frac{1}{2}}
    \label{eq:unit-step-power}
\end{equation}
The power of this signal is finite. Hence,\,\textcolor{red}{this is a power signal.}

\subsubsection{Example 2: Exponential Function}
Consider an exponential function defined as:
\begin{equation}
    x(t) = e^{-at}u(t), \, \text{where} \; u(t) \;\text{is the unit step signal},\, a > 0
    \label{eq:exp-function-example}
\end{equation}
Classify this signal as an energy, power, or neither.\\
\noindent \textbf{Solution:} Let us compute the energy of this signal as:
\begin{equation}
    E = \int_{-\infty}^{\infty} {[x(t)]}^2 \dif t = \int_{0}^{\infty} {[e^{-at}]}^2 \dif t = \int_{0}^{\infty} e^{-2at} \dif t = \frac{1}{2a} < \infty
    \label{eq:exp-function-energy}
\end{equation}
Thus, \(x(t) = e^{-at}u(t)\) is an \textcolor{red}{energy signal.}

\subsubsection{Example 3: Ramp Function}
Consider a ramp function defined as:
\begin{equation}
    r(t) =
    \begin{cases}
        At & t \geq 0         \\
        0  & \text{otherwise}
    \end{cases}
    \label{eq:ramp-function-example}
\end{equation}
Classify this signal as an energy, power, or neither.\\
\noindent \textbf{Solution:} Let us compute the energy of this signal as:
\begin{equation}
    E = \int_{-\infty}^{\infty} {r(t)}^2 \dif t = \int_{-\infty}^{0} {[0]}^2 \dif t = \int_{0}^{\infty} A^2 t^2 \dif t = A^2 \frac{T^3}{3} \Bigg|_{0}^{\infty} = \infty
    \label{eq:ramp-function-energy}
\end{equation}
Since the energy of this signal is infinite, it cannot be an energy signal. Let us compute the power of this signal as:
\begin{equation}
    P = \lim_{T \to \infty} \frac{1}{T} \int_{-T/2}^{T/2} {[r(t)]}^2 \dif t = \lim_{T \to \infty} \frac{1}{T} \int_{0}^{T/2} A^2 t^2 \dif t = A^2 \lim_{T \to \infty} \frac{1}{T} \frac{T^3}{3} \Bigg|_{0}^{\infty} = \infty
    \label{eq:ramp-function-power}
\end{equation}
The power of this signal is infinite. Hence, this is \textcolor{red}{neither a power nor an energy signal.}
\end{document}