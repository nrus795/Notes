% !TeX root = ..\..\notes-main.tex
\documentclass[../notes-main.tex]{subfiles}
\begin{document}



\section{Introduction to Modeling}
\begin{figure}[H]
    \centering
    \begin{mdframed}
        \begin{center}
            \redtext{\emph{Common man} observes the physical phenomena around him; the \emph{scientist} encodes them into models.}
        \end{center}
    \end{mdframed}
\vspace{-1em}\label{fig:label}
\end{figure}


\section{Definition of a Model}
\begin{itemize}
    \item A model of a system is a tool we use to answer questions about the system without having to do an experiment.
    \item A collection of mathematical relationships between system/process variables which purports to describe the behavior of a physical system.
    \item A convenient surrogate of the physical system.
\end{itemize}

\section{Types of Models and Their Significance}
\begin{itemize}
    \item To investigate system response under various input conditions both rapidly, and inexpensively, without tampering with the actual physical entity.
    \item Analytically design controllers.
\end{itemize}

\section{Basic Concepts: Poles, Zeros, and System Order}
\textbf{Poles and Zeros}
\begin{itemize}
    \item Physical phenomena can be categorized essentially into:
    \begin{enumerate}
        \item Energy dissipation
        \item Energy absorption or storage
    \end{enumerate}
    \item Familiar models: Resistance, Capacitance, and Inductance
    \begin{itemize}
        \item Resistance models the phenomenon of energy dissipation.
        \item Capacitance and Inductance model the phenomenon of energy absorption or storage.
    \end{itemize}
\end{itemize}

\textbf{System Order}
\begin{itemize}
    \item The order of a system equals the number of independent energy-storing elements of the system.
    \item Example:
    \begin{itemize}
        \item R-Circuit: No. of Energy storing elements = 0 Order = 0
        \item R-L Circuit: No. of Energy storing elements = 1 Order = 1
        \item R-C Circuit: No. of Energy storing elements = 1 Order = 1
        \item R-L-C Circuit: No. of Energy storing elements = 2 Order = 2
    \end{itemize}
\end{itemize}

\end{document}
