\documentclass[12pt]{report}

% Preamble: Packages and custom settings
\usepackage[utf8]{inputenc} % Allows UTF-8 encoding
\usepackage[T1]{fontenc}    % Uses modern font encoding
\usepackage{amsmath, amssymb} % Math symbols and environments
\usepackage{graphicx}       % For including images
\usepackage{hyperref}       % For clickable links and cross-references
\usepackage{enumitem}       % Customized lists
\usepackage{geometry}       % Control page margins
\usepackage{fancyhdr}       % Customize headers and footers
\usepackage{xcolor}         % Color text, if needed
\usepackage{tikz}           % For diagrams and graphics
\usepackage{circuitikz}     % For circuits!
\usepackage{booktabs}       % Improved tables

\geometry{left=2.5cm,right=2.5cm,top=2.5cm,bottom=2.5cm}

\pagestyle{fancy}
\fancyhead[L]{Nicholas Russell}
\fancyhead[C]{ELECTENG 332 Notes}
\fancyhead[R]{\today}
\fancyfoot[C]{\thepage}

\title{ELECTENG 332 Notes}
\author{Nicholas Russell}
\date{\today}

\begin{document}

\maketitle
\tableofcontents
\newpage
$$ (x+1)(x-1) $$
\chapter{Module 1: [Module Name]}
\section{Introduction}
Provide a brief introduction to this module.

\section{Topic 1: [Topic Name]}
Discuss the first major topic in this module.

\subsection{Subtopic 1.1}
Go into more detail about the first subtopic.

\subsection{Subtopic 1.2}
Further details on another subtopic.

\section{Topic 2: [Another Topic Name]}
Discuss another major topic in this module.

\clearpage

\chapter{Module 2: [Module Name]}
\section{Introduction}
Briefly introduce the second module.

\section{Topic 1: [Topic Name]}
Start with the first major topic for this module.

\subsection{Subtopic 2.1}
Detailed explanation of the first subtopic.

\subsection{Subtopic 2.2}
Further explanation or examples.

\section{Topic 2: [Another Topic Name]}
Continue with additional content for this module.

% Continue adding more chapters as needed...

\end{document}
